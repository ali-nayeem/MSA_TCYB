\section{Discussion \& Conclusion}
\label{sec:discussion}
In this study, we have introduced an application-aware multi-objective formulation to compute MSAs with an ultimate goal to infer the phylogenetic tree from the resultant alignments. To optimize MSA, we proposed two simple objective functions in addition to the existing ones. We judged the potential capability of each objective function to yield better trees by employing domain knowledge as well as by applying statistical approaches. We employed multiple linear regression to measure the degree of association between the individual objective functions and the quality of the inferred phylogenetic tree (i.e., FN rate). Thus, we provide empirical justification to choose two multi-objective formulations to move forward. Afterwards, we performed extensive experimentation with both simulated and biological datasets to demonstrate the benefit of our approach. We showed that the simultaneous optimization of a set of application-aware objective functions can lead to phylogenetic trees with improved accuracy than that of the state-of-the-art MSA tools. From this finding, we would like to hypothesize that, the use of domain specific measures can aid MSA methods in other application domain as well. In the sequel we identified \{SimG, SimNG\} to be the best set of objective functions for computing MSAs with an aim to infer phylogeny, considering its overall accuracy, runtime and nonparametric nature.

MSAs are computed to serve various biological purposes including phylogeny estimation and protein structure prediction. The definition of what constitutes a true alignment can depend partly on the purpose of MSAs~\citep{warnow2017computational}. Nevertheless, regardless of the purpose, the sites within the true alignment define the ``homologies''. Therefore, homology can be based on structural features or evolutionary histories, leading to the opposing concepts of ``structural homology'' and ``evolutionary homology''~\citep{warnow2017computational}. While structural alignments are expected to be close to the true (evolutionary) alignment, convergent evolution may create conditions where the best structural alignment puts nucleotides or amino acids in the same site (thus implying homologies), even though these specific homologies are not present in the true evolutionary alignment~\citep{iantorno2014watches}. In other words, structural homology may not be identical to evolutionary homology~\citep{reeck1987homology}. In such a situation, generic metrics such as TC/SP score might not be adequate to assess the correctness of the estimated MSAs. Therefore, using more informative metrics (e.g. phylogeny as done in the study) to tailor adequate multi-objective formulation of this problem seems a promising endeavor.

Standard criteria (SP score, TC score, etc.) for assessing alignment quality are usually based on shared homology pairs (SP score) or identical columns (TC score), and do not explicitly consider a particular application domain. Mistakes in alignments that are not important with respect to an application domain may not impact the ultimate accuracy of that particular inference. For example, not all sites are significant with respect to protein structure and function prediction, and hence multiple alignments with different accuracy may lead to the same predictions~\citep{warnow2013large}. Similarly, in the context of phylogeny estimation, alignments with substantially different SP scores may lead to trees with the same accuracy~\citep{liu2009rapid}. In this study, we systematically investigate the impact of evaluation criteria of an alignment on phylogenetic tree inference problem. Our results suggest that it could be possible to develop improved MSA methods for phylogenetic analysis by carefully choosing appropriate objective functions. Moreover, in almost all existing studies on MSA, we find the researchers evaluating the effectiveness of MSA methods using some generic alignment quality measures (i.e., TC score, SP score). Contrastingly, our results revealed that optimizing those widely used measures do not necessarily lead us to the best phylogenetic tree. This finding could be an eye opener for the researchers who need to use MSA methods to address a particular application. 

Our findings and proposed multi-objective formulation can be particularly beneficial for iterative methods like SAT\'e and PASTA that iteratively co-estimate both alignment and tree. These methods obtain an initial alignment and a tree that guide each other to improved estimates in an iterative fashion. They make an effort to exploit the close association between the accuracy of an MSA and the corresponding tree in finding the output through multiple iterations from both directions. Therefore, carefully choosing an evaluation metric for an MSA with a better correlation to the tree accuracy seems likely to improve the results of these co-estimation techniques. Thus, our methodology, if adopted, may potentially have a profound positive impact on the accuracy of these iterative co-estimation techniques. Moreover, multiple ``good'' alignments from the output of the multi-objective approach can be served as alternative MSAs for several methods (such as~\citep{ashkenazy2018multiple}) which would then utilize all of them to infer phylogeny with better accuracy.

This study will encourage the scientific community to investigate various application-aware measures for computing and evaluating MSAs. This will potentially prompt more experimental studies addressing specific application domains; and ultimately will propel our understanding of MSAs and their impact in various domains in computational biology, i.e, phylogeny estimation, protein structure and function prediction, orthology prediction etc. This study will also encourage the researchers to develop new scalable MSA tools by simultaneously optimizing multiple appropriate optimization criteria. Thus, we believe that this study will pioneer new models and optimization criteria for computing MSAs -- laying a firm, broad foundation for application specific multi-objective formulation for estimating multiple sequence alignment.

We performed an extensive experimental study comprising 29 datasets of varying sizes and complexities, and our findings are consistent throughout all the datasets. Still, we acknowledge the possibility of facing a few unforeseen circumstances as follows. There might be some datasets on which our approach might not exhibit satisfactory performance. Besides, currently we did not pay any effort to improve the running time of our approach which is higher as compared to top MSA tools. However, sufficient speedup could be achieved by leveraging modern computing architectures (computer cluster, GPU, etc.). 

Formulating application-aware multi-objective formulation (application specific evaluation criteria in general) cannot be developed entirely in one study; it should evolve in response to scientific findings and systematists' feedback. This requires the active involvement of evolutionary biologists, computer scientists, systematists, and others -- leading to improved understandings of alignments and how they are related to various fields in comparative genomics.
