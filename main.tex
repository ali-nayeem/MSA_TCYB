\documentclass[journal]{IEEEtran}
\usepackage[cmex10]{amsmath}
\usepackage{cite}
\usepackage{amsmath,amssymb,amsfonts}
\usepackage{algorithmic}
\usepackage{graphicx}
\usepackage{textcomp}
%\usepackage{natbib}
\usepackage{comment}
\usepackage{changepage}
\usepackage{array}

%\usepackage[symbol]{footmisc}
%\usepackage{mathptmx}
%\usepackage{caption}
\usepackage{subcaption}
\usepackage{hyperref}
\usepackage{verbatim}
\usepackage{float}
\usepackage{multirow}
\usepackage{xr}
\externaldocument{supp}

\usepackage{makecell}
\renewcommand\theadalign{bc}
\renewcommand\theadfont{\bfseries}
\renewcommand\theadgape{\Gape[4pt]}
\renewcommand\cellgape{\Gape[4pt]}

%\renewcommand{\thefootnote}{\fnsymbol{footnote}}
\newcommand{\commentA}[1]{{ #1}} %\color{red}
\newcommand{\commentM}[1]{{ #1}} %\color{blue} 
\newcommand{\citep}[1]{\cite{#1}}
\newcommand{\citealp}[1]{\cite{#1}}
%%%%%%%for sqeezing table column from toufiue

\newcolumntype{L}[1]{>{\raggedright\let\newline\\\arraybackslash\hspace{0pt}}m{#1}}
\newcolumntype{C}[1]{>{\centering\let\newline\\\arraybackslash\hspace{0pt}}m{#1}}
\newcolumntype{R}[1]{>{\raggedleft\let\newline\\\arraybackslash\hspace{0pt}}m{#1}}

%opening

\begin{document}
\title{Multi-objective formulation of MSA for phylogeny estimation 
	\\ \textit{\normalsize{(Do application-aware measures guide towards better phylogenetic tree?)}} }
\author{Muhammad Ali Nayeem,
	Md. Shamsuzzoha Bayzid,
	Atif Hasan Rahman,
	Rifat Shahriyar, and M. Sohel Rahman, \textit{Senior Member, IEEE}
\thanks{All the authors are with the Department of Computer Science and Engineering, Bangladesh University of Engineering and Technology, Dhaka 1205, Bangladesh. Corresponding author: M. Sohel Rahman (e-mail: msrahman@cse.buet.ac.bd).}
}
%\IEEEmembership{Senior Member, IEEE}}
%\address[1]{Department of CSE, BUET, Dhaka 1205, Bangladesh}
%\tfootnote{The first author is a recipient of the ICT PhD Fellowship administered by ICT Division, Government of People's Republic of Bangladesh.}


% The paper headers
\markboth{IEEE TRANSACTIONS ON CYBERNETICS}%
{Nayeem \MakeLowercase{\textit{et al.}}: Multi-objective formulation of MSA for phylogeny estimation}

%\corresp{Corresponding author: M. Sohel Rahman (e-mail: msrahman@cse.buet.ac.bd).}
\maketitle

\begin{abstract}
Multiple sequence alignment (MSA) is a basic step in many analyses in computational biology, including predicting the structure and function of proteins, orthology prediction and estimating phylogenies. The objective of MSA is to infer the homology among the sequences of chosen species. Commonly, the MSAs are inferred by optimizing a single function or objective. The alignments estimated under one criterion may be different from the alignments generated by other criteria, inferring discordant homologies and thus leading to different evolutionary histories relating the sequences. In recent past, researchers have advocated for the multi-objective formulation of MSA, to address this issue, where multiple conflicting objective functions are being optimized simultaneously to generate a set of alignments. However, no theoretical or empirical justification with respect to a real-life application has been shown for a particular multi-objective formulation. In this study, we investigate the impact of multi-objective formulation in the context of phylogenetic tree estimation. Employing multi-objective metaheuristics, we demonstrate that trees estimated on the alignments generated by multi-objective formulation are substantially better than the trees estimated on the alignments generated by the state-of-the-art MSA tools, including PASTA, MUSCLE, CLUSTAL, MAFFT, etc. We also demonstrate that highly accurate alignments with respect to popular measures like sum-of-pair (SP) score and total-column (TC) score do not necessarily lead to highly accurate phylogenetic trees. Thus, in essence, we ask the question of whether an application-aware (in this case phylogeny-aware) metric can guide us in choosing appropriate multi-objective formulations that can result in better phylogeny estimation. And we answer the question affirmatively through carefully designed extensive empirical study. As a by-product, we also suggest a methodology for primary selection of a set of objective functions for a multi-objective formulation based on the association with the resulting phylogenetic tree. In the sequel, we end up proposing a set of nonparametric objective functions (i.e., \{SimG, SimNG\}) that should be used by MSA methods if the goal is to produce high-quality phylogenetic trees.
\end{abstract}

\begin{IEEEkeywords}
%Enter key words or phrases in alphabetical order, separated by commas. For a list of suggested keywords, send a blank e-mail to keywords@ieee.org or visit \underline{http://www.ieee.org/organizations/pubs/ani\_prod/keywrd98.txt}
 Bioinformatics, Computational Biology, Evolutionary algorithm, Metaheuristics, Multiple sequence alignment, Multi-objective optimization, Phylogenetic tree.  
\end{IEEEkeywords}

%\titlepgskip=-15pt



\section{Introduction}
\label{sec:introducntion}
In biological research, multiple sequence alignment (MSA) is a useful and/or essential task in various applications such as phylogeny estimation, prediction of the structure and function of a RNA or protein, identification of functionally important sites, orthologous gene identification etc. The MSA task seeks to arrange more than two biological sequences to infer homology, based on certain criteria such as evolutionary history, 3D structure etc. The output is a matrix in which the input sequences are the rows and each column (i.e., site) has letters (i.e., nucleotides or amino acids) which are homologous which means all those letters descend from the same letter of a common ancestor). The aligned sequences reflect historical substitution, insertion and deletion of genetic materials which are represented as gaps. Accurately recovering these properties through MSA is necessary to accomplish a biological objective such as inferring the evolutionary history relating the sequences known as phylogenetic trees. While computing MSAs, various computational methods and criteria are used to make hypotheses about homology. But the goal of MSA is entirely biological. Figure~\ref{fig:msa_io} illustrates this problem using a hypothetical example where four protein sequences of varying lengths need to be simultaneously aligned by inserting ``appropriate'' gaps to identify homology. In this research, we limit our focus on MSA in the context of phylogeny. Phylogeny estimation from molecular sequences generally operates as a two-phase approach. At first the given sequences are aligned using an MSA method, and then a tree is estimated from the resultant alignment. The quality of inferred trees heavily depends on the quality of the corresponding alignment. There is a large body of literature in the biological domain about the relationships between multiple sequence alignments and phylogenetic trees which laid the background for this study. For example, we find several studies~\cite{jordan2011effects, chang2014tcs, lake1991order, croan1997evolution, ogden2006multiple, wu2012accounting} analyzing the effects of alignment errors and uncertainty on the accuracy of phylogenetic tree reconstruction. Therefore, it is important to select an MSA tool that is the ``most suitable'' in the phylogenetic context.

\begin{figure}[!htbp]
	%\centering
	\begin{adjustwidth}{-0.2cm}{-0.2cm}
		\includegraphics[width=0.5\textwidth]{Figure/msa_io}
		%\vspace{-0.6cm}
		\caption{A hypothetical example of an MSA problem.} 
		\label{fig:msa_io}
	\end{adjustwidth}
\end{figure}

In this study, we aim to design a multi-objective formulation of MSA that is more effective in phylogeny estimation. Previously several researchers (\citep{redelings2005joint, ashkenazy2018multiple}) made an attempt to improve the phylogeny estimation focusing on MSA computation. While they also agree with our core idea that the nature of MSA computation may influence the outputs (in a domain specific manner), they do not focus on any application-aware multi-objective formulation. %worked on the same core idea and provided solid proof of concept. We differ from them by adopting the multi-objective approach.
Our motivation for a multi-objective formulation comes from the fact that the alignment estimated under one objective may be different to the alignments generated under other objectives, inferring discordant homologies and thus leading to different and often conflicting evolutionary histories relating the sequences under consideration. Multi-objective formulations can address this issue by optimizing multiple conflicting objectives simultaneously to generate a set of alignments. However, we are faced with the challenge of using appropriate measures/metrics to choose from among a number of objective sets to optimize. So, we ask the natural question whether the popular general purpose measures to judge the alignment quality can truly reflect the quality in the context of a particular application domain, i.e., phylogeny estimation in our case. While this question has received some shallow discussion in several studies (\citep{warnow2013large, mirarab2015pasta, liu2009rapid}), to the best of our knowledge no systematic investigation has been reported in the literature to this end. Therefore, in essence, we systematically investigate whether an application-aware metric can guide us better in choosing appropriate multi-objective formulation or tools capable of generating alignments that can produce better phylogenetic trees.


There are numerous tools available in the literature to compute MSAs. We can broadly divide them into three groups: progressive techniques, consistency based techniques and iterative techniques. This division is not exclusive as many tools also use a combination of these techniques. Progressive technique is the foundation of many MSA tools such as, Clustal $\Omega$~\citep{sievers2011fast}, PRANK~\citep{loytynoja2005algorithm}, Kalign~\citep{lassmann2008kalign2}, FSA~\citep{bradley2009fast}, RetAlign~\citep{szabo2010reticular} etc. They compute the alignment using a guide tree by aligning pairs of sequences in a ``bottom-up'' manner. Among them, FSA employs an explicit statistical model to generate the alignments. It is the only method that gives an estimation of uncertainty for every column and character of the alignment. Moreover, it utilizes machine-learning techniques to estimate gap and substitution parameters at runtime for each input data. 

The consistency based techniques first construct a database of local and global pairwise alignments to facilitate generating an overall accurate alignment. The representatives of this category are T-Coffee~\citep{notredame2000t}, ProbCons~\citep{do2005probcons}, MSAProb~\citep{liu2010msaprobs}, ProbAlign~\citep{roshan2006probalign} etc. On the other hand, the iterative techniques were designed to achieve reliable alignments. These techniques try to fix the effect of mistakes made during the initial phases by repeating some crucial steps until some criteria are met. We find several examples of such techniques, such as, MAFFT~\citep{katoh2002mafft}, MUSCLE~\citep{edgar2004muscle}, MUMMALS~\citep{pei2006mummals}, ProbCons%, PRIME~\citep{yamada2006improvement}, SAGA~\citep{notredame1996saga} 
etc. In this category, we also see some ``meta-methods'' such as, SAT\'e~\citep{liu2009rapid} and PASTA~\citep{mirarab2015pasta}, which co-estimate alignment and tree using other methods. These tools achieve scalability by employing the divide-and-conquer principle and are being used widely in practice.

The performance of an MSA tool is usually evaluated by comparing its output alignment with the reference alignment (provided with the dataset as the ground truth) in terms of several measures. To this end, the most popular measures are perhaps sum-of-pair (SP) score and total-column (TC) score. SP score is the fraction of the homologies (i.e., pairs of aligned characters) in the reference alignments recovered in the estimated alignment. Similarly, TC score is the fraction of the actual aligned columns that appear in the estimated alignment.
%; both of these measures are well-known basics in MSA literature and will be defined shortly in a subsequent section. 

In this post-genomic era, the MSA datasets are posing new challenges to the researchers. Usually, an MSA method is provided with a default parameter configuration for aligning any problem instance with satisfactory accuracy. But we know that these default values can not guarantee the best output throughout all kinds of datasets~\citep{rubio2018characteristic}. For instance, there is a parameter in ProbCons called the number of iterative refinement passes. Although it can vary between 0 to 1000 the default value is set to 100. We can achieve better results by tuning the parameter values which is not a straightforward task. A systematic approach, namely, parameter advertising~\citep{deblasio2015parameter}, helps to choose the best parameter setting of an MSA method for each input data. Moreover, despite rigorous parameter tuning, no method can consistently outperform other methods for all datasets. 

Therefore, we see the emergence of novel approaches that combine different alignment tools \citep{thompson2011comprehensive}. One such approach is metaheuristics where alignments generated from different tools are exploited to produce improved alignments without vesting any effort in parameter tuning. The success of such a metaheuristic approach depends on the selection of proper objective function that can help to select better solutions from among the alternatives and thereby guide the search process towards optimal solutions of MSA.
%can push the solutions (i.e., alignments) towards the desired zone that reflects the actual purpose of an alignment task. 
As any single objective function alone can not be effective to tackle different challenges, it is wise to simultaneously optimize multiple objective functions. This will produce a set of competing solutions as the final output, which can be expected to contain our desired solution(s). Thus the formulation of MSA as a multi-objective optimization problem turns out to be appealing.

During the last decade, we find several studies (\citep{da2010alineaga, ortuno2013optimizing, soto2014multi, abbasi2015local, rubio2016hybrid,zambrano2017comparing}) with multi-objective formulation for MSA have been published -- proposing two to four objective functions to capture and quantify different aspects of an alignment. Among them, probably the most popular is the sum-of-pairs score and its weighted variants, where pairwise score is calculated for each pair of aligned sequences using a substitution matrix. This matrix should reflect the characteristics of the data at hand. Although we know that the same character across all rows of a column does not necessarily indicate homology, the count of such columns in an alignment is seen as a maximization objective known as totally conserved columns. Next, we find attempts to minimize total number of gaps to maintain the compactness of an alignment. Then there are different types of gap penalties that penalize each sequence for introducing gaps. Also we find two other objective functions, Entropy and Similarity, that compute column-wise scores and then sum those together. Both of them try to express the homogeneity of characters in a column using two different ways. Contrary to the performance/quality measures mentioned earlier (such as SP score, TC score), we are not allowed to use the reference alignment while calculating these objective functions. 

We notice several \commentA{issues} in the works advocating multi-objective formulation of MSA (in the context of different applications where the MSA will be used). First of all, in these works, there is a lack of sound theoretical or empirical justification for the choice of a particular objective function to be optimized. Secondly, we also notice the absence of a sound rationale/justification behind the two most popular performance metrics, namely, sum-of-pair score and total column score. On the contrary, it seems only natural that performance score should reflect the actual purpose of MSA. For example, if the goal is to estimate a phylogenetic tree, the performance metric to be used for evaluation should be able to accurately measure the quality and usefulness of the constructed tree. Notably, another \commentA{issue}, specific to the domain of phylogeny estimation, is the use of relatively smaller (number of taxa below 50) datasets in experiments. 
%Moreover, those works used relatively smaller datasets with number of taxa being less than 50. 

In this article, we attempt to demonstrate the effectiveness of multi-objective MSA by addressing the above mentioned limitations in the context of its intended application domain (i.e., phylogeny estimation).
%We choose a multi-objective formulation from existing studies based on their potential to produce good phylogenetic trees. Afterwards, we propose a new formulation following a similar trend. 
To make a fair comparison with nine state-of-the-art MSA tools, we conduct comprehensive experimentations on both simulated and biological datasets using tree as well as alignment quality measures. 
%We make our source code public at~\url{https://github.com/ali-nayeem/MSA}. 
Our study represents the only known work on devising a phylogeny-aware multi-objective formulation for MSA. In particular, this article makes the following key contributions:
\begin{itemize}
	\item To the best of our knowledge, this is the first attempt to investigate whether a domain specific measure (as opposed to generic alignment measures) can guide us better in choosing an appropriate multi-objective formulation or tool for MSA when the goal is to infer phylogeny. Notably, although there exist some prior works that proposed different approaches in MSA computation (such as averaging MSA~\citep{ashkenazy2018multiple}) to improve phylogeny reconstruction, our novelty lies in providing an application-aware multi-objective formulation for MSA.
	
	\item We suggested a methodology based on multiple linear regression to judge the potential efficacy of a multi-objective formulation of MSA. Then, based on this methodology, we proposed two multi-objective formulations that had the potential to yield better phylogenetic trees.
	
	\item Finally, we demonstrated that the multi-objective formulations can consistently yield better phylogenetic trees than several state-of-the-art MSA tools. Following our methodology, we identified \{SimG, SimNG\} to be the best set of objective functions for computing MSAs with an aim to infer phylogeny, considering its overall accuracy, runtime and nonparametric nature. And, interestingly we found that popular alignment quality measures do not necessarily lead to highly accurate phylogenetic trees. 
	
	\comment{ 
		\item \underline{Here starts the old contributions:} \\We first studied a phylogeny-aware multi-objective MSA formulation by examining how an objective function associates with the phylogenetic tree quality.
		\item We demonstrated that, the multi-objective optimization of MSA can consistently outperform several state-of-the-art MSA tools.
		\item We conducted a series of statistical tests to establish the significance of differences between the performance (with respect to phylogenetic tree) of multi-objective formulation and the MSA tools.
		\item We empirically show that, optimizing widely used alignment quality measures may not lead us to better phylogenetic trees.
		\item Our study revealed two simple and non-parametric (which does not depend on the characteristics of the dataset) objective functions which are effective for phylogeny estimation. 
	}
\end{itemize}
  
\section{Methods}
\label{sec:methods}
We begin this section with an overview of our experimental design. Then we introduce the objective functions that we used to compute MSA. Finally, we discuss the multi-objective metaheuristics applied to optimize those objective functions as well as the state-of-the-art tools that we utilized in this study. Because we deal with a number of objective functions, more than 10 state-of-the-art MSA tools and more than 30 instances of MSA problem, a reader is exposed to an over-preponderance of acronyms and short-cut notations. Therefore, for the sake of ease in exposition and understanding, we alphabetically list all the acronyms used in this study with their usage in Table~\ref{tab:acronyms} of the supplementary file.
%Afterwards, we describe our method of evaluating estimated alignments and phylogenetic trees. Finally, we summarize the methodology followed to establish the effectiveness of our objectives. 

\begin{figure}[!htbp]
	\centering
	%\begin{adjustwidth}{-0.5cm}{-0.5cm}
	\includegraphics[width=0.5\textwidth]{Figure/pipeline}
	\caption{Our methodology for finding the impact of a multi-objective formulation (i.e., a set of objective functions) of MSA on phylogenetic tree estimation.} 
		%For each dataset (i.e., unaligned sequences), we run a multi-objective metaheuristic. It simultaneously optimizes the given objective functions and outputs a set of alignments which represents the best-possible compromise among all objective functions. We also run several existing MSA tools on that dataset and each tool generates one alignment. We evaluate the quality of each generated alignments with respect to the reference alignments using widely used scores. Also, we estimate phylogenetic trees for all alignments and evaluate each tree with respect to the reference. Then we compare the alignments and the corresponding phylogenetic trees generated by the multi-objective formulation with the ones generated by the existing tools based on alignment score as well as tree quality. We also observe the association between alignment scores and tree quality values to examine whether it is appropriate to use alignment score in the context of phylogeny estimation.
	\label{fig:pipeline}
	%\end{adjustwidth}
\end{figure}

\subsection{Experimental design}
Our experimental methodology is briefly described below (please see also Figure~\ref{fig:pipeline}):
\begin{itemize}
	\item \underline{Step 1:} Following a systematic approach involving multiple linear regression applied on a simulated dataset, we first make an attempt to identify and choose two multi-objective formulations that turn out to be potentially more effective in the context of phylogeny estimation (discussed in Section~\ref{sec:obj_eval} of the supplementary file). 
	\item \underline{Step 2:} We run a popular and effective multi-objective metaheuristics on biological datasets to optimize each set of objective functions selected in Step 1. Each run of the metaheuristics on each dataset gives us a set of alignments as the final output. 
	\item \underline{Step 3:} We also run nine state-of-the-art MSA tools (please see Table~\ref{tab:msa_tools}) to generate alignments on all these datasets.
	% \comment{with default parameters. It is the usual practice of MSA literature. Should I exclude this?} and then estimate maximum likelihood (ML) phylogenetic tree from those alignments. We evaluate the quality of these alignments and ML trees.
	\item \underline{Step 4:} We evaluate the quality of each generated alignment with respect to the reference alignment using two popular measures, namely, SP score and TC score (discussed in Section~\ref{sec:msa_eval} of the supplementary file). 
	%For biological datasets, we used manually curated alignments as reference.
	\item \underline{Step 5:} For each of the generated alignments, we infer maximum likelihood (ML) phylogenetic tree (discussed in Section~\ref{sec:tree_estimation} of the supplementary file). Then we measure the quality of each inferred tree with respect to the reference tree (true tree) using a commonly used measure in the literature called false negative (FN) rate~\citep{warnow2017computational} (discussed in Section~\ref{sec:tree_eval} of the supplementary file).
	\item \underline{Step 6:} Finally we compare the alignments and the corresponding ML trees generated by the multi-objective optimization with the ones generated by the state-of-the-art tools. % using statistical measures.
\end{itemize}

\subsection{Objective functions}
\label{sec:formulation}
Most real-world optimization problems naturally work towards achieving several objectives. Some of these objectives are conflicting to each other. However, these problems can be transformed into single-objective ones using various simplifying techniques to avoid complexities~\citep{kalyanmoy2001multi}. On the contrary, a multi-objective formulation defines the problem using a set of objective functions and subsequently specialized methods can be applied to optimize all the objectives simultaneously. In this study, we have selected the following three multi-objective formulations of MSA from the literature based on their simplicity as well as performance as reported in the literature.
\begin{itemize}
	\item \{SOP, TC\}: Maximize the sum of pairs (SOP) and the number of totally aligned columns (TC)~\citep{da2010alineaga}.
	
	\item \{Gap, SOP\}: Maximize the sum of pairs (SOP) and minimize the number of gaps (Gap)~\citep{abbasi2015local}.
	
	\item \{wSOP, TC\}: Maximize the weighted sum of pairs with affine gap penalties (wSOP) and the number totally aligned columns (TC)~\citep{rubio2016hybrid}.
\end{itemize}

We describe these objective functions along with several existing ones in Section~\ref{sec:objective _function} of the supplementary file. Now, we propose four new objective functions that quantify different aspects of an MSA. Unlike the existing objective functions in the literature, we avoid combining multiple aspects of an MSA into a single objective. We introduce them as follows: 

\begin{itemize}
	
	\item \textbf{Minimize entropy (Entropy)}: We modify the usual definition by considering only non-gap column for the calculation of entropy.
	
	\item \textbf{Maximize similarity based on gap containing columns (SimG)}: Here we calculate similarity only for those columns that contain at least one gap. 
	
	\item \textbf{Maximize similarity based on non-gap columns (SimNG)}: We consider only non-gap column while measuring similarity.
	
	\item \textbf{Maximize concentration of gaps (GapCon)}: We find that a widely used objective function, namely, affine gap penalty~\citep{rani2016multiple}, combines two aspects of an aligned sequence, number of gaps and concentration of gaps, into a single one using weighted sum. We need to tune the weight values based on the dataset. To avoid this tuning we decide to decouple the two components. We have already considered the number of gaps as an objective function. Now we define the concentration of gaps as an independent objective which as calculated as follows. For each sequence, we count the number of consecutive gaps and take the mean of these counts. Finally, we average the resultant means for all sequences.
\end{itemize}

In this study, we used the terms shown in Table~\ref{tab:abbr} to refer to these objective functions. 

% Table generated by Excel2LaTeX from sheet 'abbr'
\begin{table}[!htbp]
	\centering
	%\small
	\caption{Terms used to denote the objective functions.}
	\begin{tabular}{|l|L{6cm}|c|} %{5cm}
		\hline
		\# & \multicolumn{1}{c|}{Objective function} & Term \\
		\hline
		1 & Maximize no. of totally aligned columns & TC \\
		\hline
		2 & Minimize no. of gaps & Gap \\
		\hline
		3 & Maximize sum of pairs & SOP \\
		\hline
		4 & Maximize weighted sum of pairs with affine gap penalties & wSOP \\
		\hline
		5 & Minimize entropy & Entropy \\
		\hline
		6 & Maximize similarity based on gap containing columns & SimG \\
		\hline
		7 & Maximize similarity based on non-gap columns & SimNG \\
		\hline
		8 & Maximize concentration of gaps & GapCon \\
		\hline
	\end{tabular}%
	\label{tab:abbr}%
\end{table}%

\begin{comment}
So, in this study we work run our optimization algorithm with these set of objective function:

\begin{itemize}
\item \{Gap, SOP\}

\item \{SOP, TC\}

\item \{wSOP, TC\}

\item \{Gap, SOP, wSOP, TC\}
\item \{Entropy, TC, Gap, SimG, SimNG, GapCon\}
\item \{SimG, SimNG\}
\end{itemize}
\end{comment}

We need a substitution matrix to calculate SOP and wSOP. The values of this substitution matrix depend on the trait of a particular dataset. In this study, we used NUC4.4 (supplied by NCBI at \url{ftp://ftp.ncbi.nih.gov/blast/matrices/NUC.4.4}) for nucleotide sequences and BLOSUM62~\citep{henikoff1992amino} for protein sequences. On the contrary, the four objective functions that we proposed are nonparametric which are independent of the dataset.

\subsection{Multi-objective metaheuristics} %Multi-objective Metaheuristics Generation of competitive alignments
To simultaneously optimize multiple objective functions, we ran two popular multi-objective metaheuristics: NSGA-II~\citep{deb2002fast} and NSGA-III~\citep{deb2014evolutionary}. They belong to the class of multi-objective evolutionary algorithms. They start from a set of candidate solutions (termed as population) and then uses mechanisms inspired by biological evolution (such as mutation, crossover, selection, etc.) to evolve the population towards the optimal solutions. Unlike single-objective optimization methods, they output a set of solutions (i.e. members of the final population)
which represents the best possible compromise of all objectives under consideration. Two studies (\citep{zambrano2017m2align, ortuno2013optimizing}) demonstrated the strength of NSGA-II for solving MSA. NSGA-II works best when the number of objectives is upto three while NSGA-III is specially designed for handling more than three objectives. Hence, we applied these algorithms according to Table~\ref{tab:variants}.
%In this study, we simultaneously optimize two to six objective functions. Therefore, we select these two algorithms to work with. 
We discuss these methods along with their vital components and parameters in Section~\ref{sec:mop} of the supplementary file. We implemented them using jMetalMSA~\citealp{zambrano2017multi} which is a Java metaheuristic framework for MSA. Our implementation is publicly available at \url{https://github.com/ali-nayeem/MSA}.
%Table~\ref{tab:variants}. 
%We implement these two metaheuristics using jMetalMSA of~\citealp{zambrano2017multi} which is a Java metaheuristic framework for MSA publicly available at~\url{https://github.com/jMetal/jMetalMSA}. 
% Table generated by Excel2LaTeX from sheet 'multi-pc'
\begin{table}[!htbp]
	%\small
	\centering
	\caption{Our selected algorithms and corresponding objective set.}
	\begin{tabular}{|c|l|} %{5cm}
		\hline
		Algorithm & Objective set \\
		\hline
		\multicolumn{1}{|c|}{\multirow{4}{*}{NSGA-II}} & \{Gap, SOP\}\\
		& \{SOP, TC\}          \\
		& \{wSOP, TC\} \\
		& \{SimG, SimNG\} \\
		\hline
		\multicolumn{1}{|c|}{\multirow{3}{*}{NSGA-III}} & \{Gap, SOP, wSOP, TC\} \\
		& \{Entropy, TC, Gap, SimG, SimNG, GapCon\} \\
		\hline
	\end{tabular}%
	\label{tab:variants}%
\end{table}%


\subsection{State-of-the-art MSA tools}
We used the alignments generated by nine representative state-of-the-art MSA tools (shown in Table~\ref{tab:msa_tools}) to compare with our approach. We run each of them with its default parameter configuration. Moreover, we initialize the multi-objective metaheuristics with a set of alignments generated by randomly mixing and modifying those nine alignments. Notably, this approach, known as the seeded initial population generation, is quite common in the metaheuristics literature specially for multi-objective optimization. 
%Table~\ref{tab:msa_tools} and Table~\ref{tab:msa_tools} show the list of the MSA tools

% Table generated by Excel2LaTeX from sheet 'single'
\begin{table}[htbp]
	%\small
	\centering
	\caption{List of state-of-the-art MSA tools that we used in this study.}
	\begin{tabular}{|l|l||l|l|}
		\hline
		\multicolumn{2}{|c||}{For nucleotide sequences} & \multicolumn{2}{c|}{For protein sequences} \\
		\hline
		\multicolumn{1}{|c|}{Tool} & \multicolumn{1}{c||}{Version} & \multicolumn{1}{c|}{Tool} & \multicolumn{1}{c|}{Version} \\
		\hline
		FSA~\citep{bradley2009fast} & 1.15.9 & FSA   & 1.15.9 \\
		\hline
		PASTA~\citep{mirarab2015pasta} & 1.7.8 & PASTA & 1.7.8 \\
		\hline
		T-Coffee~\citep{notredame2000t} & 11.00 & T-Coffee & 11.00 \\
		\hline
		MAFFT~\citep{katoh2002mafft} & 7.31  & MAFFT & 7.245 \\
		\hline
		Clustal W~\citep{thompson1994clustal} & 2.1   & Clustal W & 2.1 \\
		\hline
		Clustal $ \Omega $~\citep{sievers2011fast} & 1.2.4 & RetAlign~\citep{szabo2010reticular} & 1.0 \\
		\hline
		MUSCLE~\citep{edgar2004muscle} & 3.8.31 & MUSCLE & 3.8.31 \\
		\hline
		PRANK~\citep{loytynoja2005algorithm} & 0.170427 & ProbCons~\citep{do2005probcons} & 1.12 \\
		\hline
		Kalign~\citep{lassmann2008kalign2} & 2.03  & Kalign & 2.04 \\
		\hline
	\end{tabular}%
	\label{tab:msa_tools}%
\end{table}%

\begin{comment}

\subsection{Evaluation of estimated alignments}
We evaluate estimated alignments with respect to reference alignment using two well-known alignment quality scores called TC score and SP score. These two scores are defined below:
\begin{itemize}
\item TC score is the ratio of the number of correctly aligned columns in the estimated alignment to the total number of aligned columns in the reference alignment. This is also known as column score.

\item SP score is the ratio of the number of aligned pairs in the estimated alignment to the total number of aligned pairs in the reference alignment.

%\item Pairs score is the mean of SP-score and Modeler. SP-Score is the ratio of the number of aligned pairs to the total number of aligned pairs in the reference alignment. And Modeler is very similar to the SP-score where we take the ratio of the number of aligned pairs to the total number of aligned pairs in the estimated alignment      
\end{itemize}
For both the measures, higher value implies better score.

\subsection{Phylogenetic tree estimation}
For each of the generated alignment we estimate the phylogenetic tree using Maximum Likelihood (ML) method. We used a popular tool named FastTree-2 developed by \citealp{price2010fasttree}. %It is publicly available at \url{http://www.microbesonline.org/fasttree/}.

\subsection{Evaluation of phylogenetic tree}
We evaluate the quality of each estimated ML tree with respect to the true phylogenetic tree using a widely used measure known as the False Negative (FN) rate. FN rate is the percentage of edges present in the true tree but missing in the estimated tree. So a small value of FN rate is desirable. %as a quality measure, 

\subsection{Evaluation of objective functions}
In the context of phylogeny estimation, a desired objective function for MSA should lead to such alignments which can produce highly accurate (having small FN rate) ML trees. Considering this fact, we try to evaluate the effectiveness of an objective function by studying how its values are associated with the corresponding FN rates. The objective function that frequently exhibits positive correlation with FN rate is predicted to be a good optimization criteria. To accomplish this, we fit multiple linear regression model to calculate the degree of association (i.e., regression coefficient) between an objective and FN rate. Then we apply t-test, with null hypothesis that there is no association, to check the significance of individual regression coefficients. It should be noted that, such regression results does not necessarily indicate the strength of an objective as an optimization criterion. However, such results can definitely be utilized as the starting point for experimentation for further validation.
%Rather, we expect an objective function, which exhibits larger value compared to others frequently, to guide the optimization algorithms better than other.
\end{comment}
\begin{comment}
\subsection{Computational time}
We ran the multi-objective metaheuristics on a server with Intel(R) Xeon(R) CPU E5-4617 @ 2.90GHz processor and 64GB of RAM. In Table~\ref{tab:time}, we give a rough estimate of the total computational time that we invested to derive our results.

\begin{table}[htbp]
\small
\centering
\caption{Computational time invested to study the impact of multi-objective formualtion of MSA.}
\begin{tabular}{|l|l|}
\hline 
\multicolumn{1}{|c|}{Dataset} & \multicolumn{1}{c|}{Total time (hours)} \\ 
\hline 
100-taxon simulated dataset &  1269.38\\ 
\hline 
Biological rRNA dataset &  311.64\\ 
\hline 
BAliBASE dataset &  45.88\\ 
\hline 
\end{tabular} 
\label{tab:time}%
\end{table}%
\end{comment}
\section{Results}
\label{sec:results}

%In this section, we discuss our experimental results on both simulated and biological datasets. We conduct extensive experiments with 10 replicates of 100-taxon simulated dataset, two biological rRNA datasets and 27 BAliBASE datasets. We begin by picking two sets of objective functions by analyzing the generated set of alignments based on resultant tree quality. Then for all datasets, we compare the alignments generated by those objective sets with nine state-of-the-art tools. 

We conducted extensive experiments with both simulated and biological datasets. We begin by carefully and systematically selecting two multi-objective formulations which are potentially useful for phylogenetic tree estimation employing NSGA-III and multiple linear regression. Next, we generate alignments through running NSGA-II as well as nine state-of-the-art MSA tools. Then we compare those alignments with respect to both generic and domain specific quality measures. In this section, we discuss our obtained results after introducing our chosen datasets. In what follows, unless otherwise specified, when we discuss the (best) results of a tool, we mean one of the above-mentioned nine tools.

\begin{comment}
\begin{figure*}[!htbp]
\centering
\begin{adjustwidth}{-0.5cm}{-0.5cm}
\includegraphics[width=1.1\textwidth]{Figure/pipeline}
\caption{Our methodology for finding the impact of a multi-objective formulation (i.e., a set of objective functions) of MSA on phylogenetic tree estimation. For each dataset (i.e., unaligned sequences), we run a multi-objective metaheuristic. It simultaneously optimizes the given objective functions and outputs a set of alignments which represents the best-possible compromise among all objective functions. We also run several existing MSA tools on that dataset and each tool generates one alignment. We evaluate the quality of each generated alignments with respect to the reference alignments using widely used scores. Also we estimate phylogenetic trees for all alignments and evaluate each tree with respect to the reference. Then we compare the alignments and the corresponding phylogenetic trees generated by the multi-objective formulation with the ones generated by the existing tools based on alignment score as well as tree quality. We also observe the association between alignment scores and tree quality values to examine whether it is appropriate to use alignment score in the context of phylogeny estimation.}
\label{fig:pipeline}
\end{adjustwidth}
\end{figure*}

\subsection{Experimental design}
Our experimental methodology is briefly described below (please see also Figure~\ref{fig:pipeline}):
\begin{itemize}
\item \underline{Step 1:} Following a systematic approach involving multiple linear regression applied on a simulated dataset, we first make an attempt to identify and choose two multi-objective formulations that turn out to be potentially more effective in the context of phylogeny estimation. 
\item \underline{Step 2:} We run a popular and effective multi-objective metaheuristics on both biological and simulated datasets to optimize each set of objective functions selected in Step 1. Each run of the metaheuristics on each dataset gives us a set of alignments as output. 
\item \underline{Step 3:} We also run nine state-of-the-art MSA tools (please see Section~\ref{sec:methods}, Table~\ref{tab:msa_tools}) to generate alignments on all these datasets.
% \comment{with default parameters. It is the usual practice of MSA literature. Should I exclude this?} and then estimate maximum likelihood (ML) phylogenetic tree from those alignments. We evaluate the quality of these alignments and ML trees.
\item \underline{Step 4:} We evaluate the quality of each generated alignment with respect to the reference alignment using two popular measures, namely, SP score and TC score. 
%For biological datasets, we used manually curated alignments as reference.
\item \underline{Step 5:} For each of the generated alignments, we infer maximum likelihood (ML) phylogenetic tree. Then we measure the quality of each inferred tree with respect to the reference tree (true tree) using the mostly used measure in the literature called false negative (FN) rate.
\item \underline{Step 6:} Finally we compare the alignments and the corresponding ML trees generated by the multi-objective optimization with the ones generated by the state-of-the-art tools. % using statistical measures.
\end{itemize}
\end{comment}

\subsection{Datasets}
%We aim to validate the effectiveness of an objective function in terms of how it can help to estimate the phylogenetic tree. So, in this context, a good objective function has good statistical evidence of leading to good trees. To conduct our study we require knowledge of the true phylogenetic tree which can be used as a reference for measuring the goodness of estimated trees. 
We studied a simulated dataset (100-taxon simulated dataset~\citep{liu2009rapid}) as well as two biological datasets (biological rRNA datasets~\citep{liu2009rapid} and BAliBASE 3.0 benchmark~\citep{thompson2005balibase}). As the simulated dataset comes with the true phylogenetic tree, we use this dataset to examine whether a multi-objective formulation of MSA is potentially application-aware and in the sequel, we select two such formulations somewhat similar to the training phase of a machine learning approach. Afterwards, we validate the effectiveness of the selected formulations against the state-of-the-art MSA tools based on biological datasets.  
% We now briefly introduce them in this subsection.   

From the 100-taxon simulated dataset, we randomly selected five replicates. And among the biological datasets, we chose two challenging ribosomal RNA datasets along with 27 random instances of the widely used BAliBASE 3.0 benchmark. Section~\ref{sec:dataset_stat} of the supplementary file provides a detailed description of these datasets. 

\begin{comment}
\subsubsection{100-taxon simulated dataset}
We used 10 (out of 20) randomly selected replicates (R0, R2, R4, R6, R9, R10, R13, R14, R17, R19) of simulated nucleotide dataset from the study of~\citealp{liu2009rapid}. It is publicly available at \url{https://sites.google.com/eng.ucsd.edu/datasets/sate-i}. Table~\ref{tab:sim_stat} in the supplementary file gives the reference alignment statistics for this dataset.
% generated from a model tree (can be used as true tree) with model condition (100 taxa, short gap length type)



% Table generated by Excel2LaTeX from sheet 'Sheet1'
\begin{table}[htbp]
\centering
\caption{Reference alignments for 100-taxon simulated dataset.}
\begin{tabular}{|l|r|}
\hline
\multicolumn{1}{|c|}{Feature} & \multicolumn{1}{c|}{Value} \\
\hline
Number of taxa & 100 \\
\hline
Number of sites & 1698.2 \\
\hline
Percent indels & 40.4 \\
\hline
Avg. gap length & 3.1 \\
\hline
\end{tabular}%
\label{tab:sim_stat}%
\end{table}%


\subsubsection{Biological rRNA datasets}
We analyzed two biological ribosomal RNA datasets, 23S.E and 23S.E.aa\_ag, from~\citealp{liu2009rapid} which are challenging for phylogeny estimation methods. Each of these datasets is given with a highly reliable, curated reference alignment from Gutell Lab. The statistics of the reference alignments of these datasets are presented in Table~\ref{tab:bio_stat} of the supplementary file. Reference trees for these datasets were generated from the reference alignments by running RAxML~\citep{stamatakis2014raxml} with bootstrapping, and retaining only the highly supported edges. We evaluated generated alignments with respect to the reference alignment using the tool FastSP \citep{mirarab2011fastsp}.

\begin{comment}

% Table generated by Excel2LaTeX from sheet 'Sheet2'
\begin{table}[htbp]
\small
\centering
\caption{Reference alignments for two biological rRNA datasets.}
\begin{tabular}{|l|r|r|}
\hline
\multicolumn{1}{|c|}{Feature} & \multicolumn{1}{c|}{23S.E.aa\_ag} & \multicolumn{1}{c|}{23S.E} \\
\hline
Number of taxa & 144   & 117 \\
\hline
Number of sites & 8,619 & 9,079 \\
\hline
Percent indels & 61.1  & 59.7 \\
\hline
Avg. gap length & 13.5  & 12.6 \\
\hline
\end{tabular}%
\label{tab:bio_stat}%
\end{table}%



\subsubsection{BAliBASE datasets}
BAliBASE 3.0 \citep{thompson2005balibase} is the most widely used benchmark alignment databases of protein families. It provides manually refined reference alignments of high quality based on 3D structural superposition. These datasets are organized into six groups according to their families and similarities: \commentA{RV11 (very divergent sequences, residue identity below 20\% ), RV12 (medium to divergent sequences, 20\%-40\% residue identity), RV20 (families with one or more highly divergent sequences), RV30 (divergent subfamilies), RV40 (sequences with large terminal N/C extensions), and RV50 (sequences with large internal insertions).} In this study, we selected four to five representative datasets from each group as reported in Table~\ref{tab:balibase} of the supplementary file. We generated reference trees for these datasets by running RAxML with bootstrapping. We evaluated estimated alignments with respect to the core blocks (regions for which reliable alignments are known to exist) using the program bali\_score available at~\url{http://www.lbgi.fr/balibase/BalibaseDownload/}.
% We randomly take 27 (out of 218) datasets (Table~\ref{tab:balibasel})

\begin{comment}

% Table generated by Excel2LaTeX from sheet 'Sheet2'
\begin{table}[htbp]
\small
\centering
\caption{ BAliBASE datasets selected for this study.}
\begin{tabular}{|l|L{5.1cm}|}
\hline
\multicolumn{1}{|c|}{Group} & \multicolumn{1}{c|}{Datasets selected} \\
\hline
RV11  & BB11005, BB11018, BB11020, BB11033 \\
\hline
RV12  & BB12001, BB12013, BB12022, BB12035, BB12044 \\
\hline
RV20  & BB20001, BB20010, BB20022, BB20033, BB20041 \\
\hline
RV30  & BB30002, BB30008, BB30015, BB30022 \\
\hline
RV40  & BB40001, BB40013, BB40025, BB40038, BB40048 \\ %
\hline
RV50  & BB50001, BB50005, BB50010, BB50016 \\
\hline
\end{tabular}%
\label{tab:balibase}%
\end{table}%
\end{comment}

\subsection{Selection of appropriate multi-objective formulations}
\label{sec:selection_msa_formulation}
%We choose two sets of objective functions (i.e. multi-objective formulation) to conduct experiments based on this dataset. We select the first set of objective functions from among three existing studies based on their relative performance in the context of phylogeny estimation. Afterwards, we follow a similar approach to form the second set by incorporating our proposed objective functions. Now we discuss the selection process of these two sets along with their performance on this dataset.
As has been mentioned above, we have used 100-taxon simulated dataset to select one or more multi-objective formulations that have the potential to be ``application-aware''. We first conduct extensive experiments to choose a formulation (i.e., a set of objective functions) of MSA from among the existing popular ones from the literature (Section~\ref{sec:existing_msa_formulation}); subsequently, we also suggest a new promising formulation (Section~\ref{sec:new_msa_formulation}).  
%We use this dataset to select a potentially application-aware multi-objective formulation (i.e., a set of objective functions) of MSA from the literature. Then based on the same dataset we present a new promising formulation. We perform these steps by adopting a novel methodology based on the careful application of multiple linear regression. Now we discuss the selection process of these two formulations followed by their performance against the state-of-the-art tools.
%At the end, we compare these two objective set based on their performance on 10 replicates.
\begin{comment}

\begin{figure*}[!htbp]    
\begin{adjustwidth}{-1cm}{-1cm}
\centering
\begin{subfigure}{0.35\textwidth}
\includegraphics[width=\columnwidth]{Figure/NumGaps_SOP_TC_wSOP/precomputedInit/R0/fig/scatter_mattrix}
\caption{R0}
%\label{fig:con_pr09}
\end{subfigure}    
\begin{subfigure}{0.35\textwidth}
\includegraphics[width=\columnwidth]{Figure/NumGaps_SOP_TC_wSOP/precomputedInit/R4/fig/scatter_mattrix}
\caption{R4}
%\label{fig:con_pr09}
\end{subfigure}
\begin{subfigure}{0.35\textwidth}
\includegraphics[width=\columnwidth]{Figure/NumGaps_SOP_TC_wSOP/precomputedInit/R9/fig/scatter_mattrix}
\caption{R9}
%\label{fig:con_pr09}
\end{subfigure}
\begin{subfigure}{0.35\textwidth}
\includegraphics[width=\columnwidth]{Figure/NumGaps_SOP_TC_wSOP/precomputedInit/R14/fig/scatter_mattrix}
\caption{R14}
%\label{fig:con_pr09}
\end{subfigure}
\begin{subfigure}{0.35\textwidth}
\includegraphics[width=\columnwidth]{Figure/NumGaps_SOP_TC_wSOP/precomputedInit/R19/fig/scatter_mattrix}
\caption{R19}
%\label{fig:con_pr09}
\end{subfigure}
\caption{\underline{100-taxon simulated dataset:} Scatter-plot matrices depicting the pairwise relationship of all objective functions on five randomly selected replicates. We turn each objective function into minimization form and then normalize using min-max technique. In each matrix, the diagonal cells show the distribution of objective values (estimated using KDE) while the non-diagonal cells show the correlation between pairs of objective functions. Each upper-diagonal cell contains the value of correlation coefficient $r$ of the corresponding pair of objective functions.}
\label{fig:nature_obj}
\end{adjustwidth}
\end{figure*}

\begin{figure*}[!htbp]
\centering
\small
\begin{adjustwidth}{-1cm}{-1cm}
\begin{tabular}{l||C{0.24\textwidth}|C{0.24\textwidth}|C{0.24\textwidth}|C{0.24\textwidth} }
& TC & Gap & SOP & wSOP\\\hline\hline
\rotatebox[origin=c]{-90}{R0} & 
\raisebox{-.5\height}{\includegraphics[width=0.25\textwidth]{Figure/NumGaps_SOP_TC_wSOP/precomputedInit/R0/fig/tc_partial_regression}} &
\raisebox{-.5\height}{\includegraphics[width=0.25\textwidth]{Figure/NumGaps_SOP_TC_wSOP/precomputedInit/R0/fig/gap_partial_regression}} & 
\raisebox{-.5\height}{\includegraphics[width=0.25\textwidth]{Figure/NumGaps_SOP_TC_wSOP/precomputedInit/R0/fig/sop_partial_regression}} & 
\raisebox{-.5\height}{\includegraphics[width=0.25\textwidth]{Figure/NumGaps_SOP_TC_wSOP/precomputedInit/R0/fig/wsop_partial_regression}}     
\\\hline
\rotatebox[origin=c]{-90}{R4} &
\raisebox{-.5\height}{\includegraphics[width=0.25\textwidth]{Figure/NumGaps_SOP_TC_wSOP/precomputedInit/R4/fig/tc_partial_regression}} &
\raisebox{-.5\height}{\includegraphics[width=0.25\textwidth]{Figure/NumGaps_SOP_TC_wSOP/precomputedInit/R4/fig/gap_partial_regression}} & 
\raisebox{-.5\height}{\includegraphics[width=0.25\textwidth]{Figure/NumGaps_SOP_TC_wSOP/precomputedInit/R4/fig/sop_partial_regression}} & 
\raisebox{-.5\height}{\includegraphics[width=0.25\textwidth]{Figure/NumGaps_SOP_TC_wSOP/precomputedInit/R4/fig/wsop_partial_regression}}
\\\hline
\rotatebox[origin=c]{-90}{R9} &
\raisebox{-.5\height}{\includegraphics[width=0.25\textwidth]{Figure/NumGaps_SOP_TC_wSOP/precomputedInit/R9/fig/tc_partial_regression}} &
\raisebox{-.5\height}{\includegraphics[width=0.25\textwidth]{Figure/NumGaps_SOP_TC_wSOP/precomputedInit/R9/fig/gap_partial_regression}} & 
\raisebox{-.5\height}{\includegraphics[width=0.25\textwidth]{Figure/NumGaps_SOP_TC_wSOP/precomputedInit/R9/fig/sop_partial_regression}} & 
\raisebox{-.5\height}{\includegraphics[width=0.25\textwidth]{Figure/NumGaps_SOP_TC_wSOP/precomputedInit/R9/fig/wsop_partial_regression}}
\\\hline
\rotatebox[origin=c]{-90}{R14} &
\raisebox{-.5\height}{\includegraphics[width=0.25\textwidth]{Figure/NumGaps_SOP_TC_wSOP/precomputedInit/R14/fig/tc_partial_regression}} &
\raisebox{-.5\height}{\includegraphics[width=0.25\textwidth]{Figure/NumGaps_SOP_TC_wSOP/precomputedInit/R14/fig/gap_partial_regression}} & 
\raisebox{-.5\height}{\includegraphics[width=0.25\textwidth]{Figure/NumGaps_SOP_TC_wSOP/precomputedInit/R14/fig/sop_partial_regression}} & 
\raisebox{-.5\height}{\includegraphics[width=0.25\textwidth]{Figure/NumGaps_SOP_TC_wSOP/precomputedInit/R14/fig/wsop_partial_regression}}
\\\hline
\rotatebox[origin=c]{-90}{R19} &
\raisebox{-.5\height}{\includegraphics[width=0.25\textwidth]{Figure/NumGaps_SOP_TC_wSOP/precomputedInit/R19/fig/tc_partial_regression}} &
\raisebox{-.5\height}{\includegraphics[width=0.25\textwidth]{Figure/NumGaps_SOP_TC_wSOP/precomputedInit/R19/fig/gap_partial_regression}} & 
\raisebox{-.5\height}{\includegraphics[width=0.25\textwidth]{Figure/NumGaps_SOP_TC_wSOP/precomputedInit/R19/fig/sop_partial_regression}} & 
\raisebox{-.5\height}{\includegraphics[width=0.25\textwidth]{Figure/NumGaps_SOP_TC_wSOP/precomputedInit/R19/fig/wsop_partial_regression}}
\\\hline
\end{tabular}
\caption{\underline{100-taxon simulated dataset:} Multiple linear regression model for identifying the association among FN rate and three objective functions (TC, Gap and SOP/wSOP) fitted to five randomly selected replicates. There is one figure for each possible combination (replicate, objective function). Each partial regression plot shows the association between an objective function and FN rate while holding the remaining two objectives constant. In a plot for an objective function $ OF $, the horizontal axis, $e(OF|X)$, denotes the residuals from regressing $OF$ against the remaining objective functions and the vertical axis, $e(FNRate|X)$, denotes the residuals from regressing FN rate against all the objective functions except $ OF $.} 
\label{fig:mul_lin_reg}
\end{adjustwidth}
\end{figure*}

\begin{figure*}[!htbp]
\centering
\begin{adjustwidth}{-1cm}{-1cm}
\begin{subfigure}{0.22\textwidth}
\includegraphics[width=\columnwidth]{Figure/summary/precomputedInit/R0/objset_fnrate_rank}
\caption{R0}
%\label{fig:con_pr09}
\end{subfigure}    
\begin{subfigure}{0.22\textwidth}
\includegraphics[width=\columnwidth]{Figure/summary/precomputedInit/R4/objset_fnrate_rank}
\caption{R4}
%\label{fig:con_pr09}
\end{subfigure}
\begin{subfigure}{0.22\textwidth}
\includegraphics[width=\columnwidth]{Figure/summary/precomputedInit/R9/objset_fnrate_rank}
\caption{R9}
%\label{fig:con_pr09}
\end{subfigure}
\begin{subfigure}{0.22\textwidth}
\includegraphics[width=\columnwidth]{Figure/summary/precomputedInit/R14/objset_fnrate_rank}
\caption{R14}
%\label{fig:con_pr09}
\end{subfigure}
\begin{subfigure}{0.22\textwidth}
\includegraphics[width=\columnwidth]{Figure/summary/precomputedInit/R19/objset_fnrate_rank}
\caption{R19}
%\label{fig:con_pr09}
\end{subfigure}
\caption{\underline{100-taxon simulated dataset:} Comparison among objective sets based on the distribution of the collection of the best FN rates from each run. The performance of the state-of-the-art tools are shown using horizontal lines.}
\label{fig:rank_best_fn_rate}
\end{adjustwidth}
\end{figure*}
\end{comment}

\subsubsection{Selection from among the existing formulations}
\label{sec:existing_msa_formulation}
To reduce the computational effort, we pre-select three multi-objective formulations of MSA and limit our investigation thereon. Thus we choose one of the formulations from among \{Gap, SOP\}~\citep{abbasi2015local}, \{SOP, TC\}~\citep{da2010alineaga} and \{wSOP, TC\}~\citep{rubio2016hybrid} (please see Section~\ref{sec:methods} Table~\ref{tab:abbr}). We experiment with five randomly selected replicates (R0, R4, R9, R14, R19) and then judge based on two criteria: firstly, we used multiple linear regression analysis to examine the association between individual objective function and FN rate; secondly, we assess the alignments generated through the optimization of each set of objective functions in terms of resultant ML trees. 

We need to consider the relationship between each pair of objective functions to properly interpret the result of multiple linear regression. We perform this by running an appropriate multi-objective metaheuristic (i.e., NSGA-III~\citep{deb2014evolutionary}) for 25 times which simultaneously optimizes all the objective functions (i.e., \{Gap, SOP, wSOP, TC\}) and thus we obtain a large collection of diverse alignments be merging the sets of solutions output by each run. 
%We visualize the interrelations among the objective values of those solutions using a $ 4\times4 $ scatter-plot matrix~\cite{kalyanmoy2001multi} as shown in Figure~\ref{fig:nature_obj}. Here each diagonal cell of a matrix depicts the distribution of the values of an objective function estimated using Kernel Density Estimation (KDE) and the non-diagonal cells show the correlation between each pair of objective functions. As our metaheuristic algorithm tries to minimize all objective functions, we treat the maximization objective values by multiplying with -1. In the sequel, we normalize all the objective values using min-max technique and as such the maximization objectives are turned into minimization ones. 
A visualization of the interrelations among the objective values of those solutions is presented in Figure~\ref{fig:nature_obj} of the supplementary file. From these experiments, we have the following two key observations.
\begin{enumerate}[(a)]
	
	\item In all the cases, SOP is totally correlated with wSOP. So we do not need to optimize both of them. Moreover, this high correlation creates a serious problem in multiple regression analysis called multicollinearity~\citep{montgomery2012introduction}. Therefore, we should not keep these two objective functions together in our regression analysis. Also, it is redundant to consider both of them in the multi-objective formulation. 
	
	\item SOP is clearly in conflict with Gap across all the replicates. Therefore, if we optimize them simultaneously, we can generate many diverse solutions which represent the compromise between these two objective functions~\citep{kalyanmoy2001multi}. This diverse collection is likely to contain the desired alignment for any kind of dataset.
	
\end{enumerate}

As the objective functions are inter-related, we need to measure the degree of association between an objective and FN rate while holding the remaining objectives constant to avoid getting any spurious result~\citep{montgomery2012introduction}. Therefore, we perform multiple linear regression by employing the following model:
\begin{equation}
\small
\begin{split}
\text{FN rate} = \beta_0 + \beta_1 \times \text{TC}+ \beta_2 \times \text{Gap} + \\
\beta_3 \times \text{SOP (or wSOP)} + \epsilon \label{eq:multi_lin_reg}
\end{split}
\end{equation}

Each coefficient ($\beta_1, \beta_2$ and $\beta_3$) represents the expected change in the FN rate per unit change in the corresponding objective function when all the remaining objective functions are held constant. For this reason, they ($\beta_i$) are called partial regression coefficients. $\epsilon$ is the random error component which is assumed to follow a Gaussian distribution with mean zero and some fixed standard deviation. We fit this model to the solutions generated by optimizing the set \{Gap, SOP, wSOP, TC\}. For each of those solutions, we estimate the ML tree and evaluate its quality in terms of FN rate.  
%We estimate these coefficients using least-squares method and illustrate them using partial regression plots~\citep{montgomery2012introduction} in Figure~\ref{fig:mul_lin_reg}. We apply $t$-test on individual regression coefficient (i.e., slope) $\beta_i$ (with null hypothesis $\beta_i=0$) to test the significance of that association. The test results (slope, $p$-value) are incorporated in the figure. We can note the following two interesting points from these results.
We estimate these coefficients using the least-squares method (an illustration is presented in Figure~\ref{fig:mul_lin_reg} of the supplementary file). We apply $t$-test on individual regression coefficient (i.e., slope) $\beta_i$ (with null hypothesis $\beta_i=0$) to test the significance of that association. We can note the following two interesting points from these results.
\begin{enumerate}[(a)]
	\item In the majority of the cases (R0, R4 and R14), Gap, SOP and wSOP exhibit a good degree of association with FN rate (i.e positive slope) with high confidence (p-value close to 0) compared to other objective functions. So, we can expect them to be good optimization objectives for MSA.
	\item For replicate R4 and R19, none of the objective exhibit good association. This shows that an objective function might not perform well across all problem instances. 
	%\item We notice that for R19, the slope directions are opposite to our expectation. And for R19, the slope directions are opposite to our expectation.  
\end{enumerate}

Now we measure the strength of each objective set based on the FN rate achieved by the members of the generated solution set. To accomplish this, for each set of objective functions, we run a suitable multi-objective metaheuristics (NSGA-II~\citep{deb2002fast}) for 20 times following the standard practice of operations research (OR) literature (due to the stochastic nature of metaheuristics). Each run generates a set of solutions that represents the trade-offs in satisfying all objectives. Afterwards, we inferred the ML tree for each of the generated alignment. We collected the best FN rates from each of the 20 solution sets and examine the distribution of these FN rates (a visualization of these distributions using boxplots is presented in Figure~\ref{fig:rank_best_fn_rate} of the supplementary file). Here we have the following key observations:
\begin{itemize}
	\item For most of the cases, the combined set \{TC, Gap, SOP, wSOP\} achieves better results than the other sets. This indicates that adding suitable objective functions increase the chance of achieving the best FN rate. However, this can increase the overall complexity of the multi-objective metaheuristic. So in this study, we keep the size of the objective set as small as possible.
	%\item For R19, where we saw an unusual regression result, the objective sets perform worse compared to other replicates.
	\item Among our three pre-selected objective sets, \{Gap, SOP\} achieves relatively lower FN rates. This is consistent with the regression results discussed earlier.
	\item Both \{TC, Gap, SOP, wSOP\} and \{Gap, SOP\} persistently yield better FN rates than the state-of-the-art tools.
\end{itemize}
Based on our findings discussed so far, we consider \{Gap, SOP\} to be the most suitable candidate to conduct our study among all the formulations considered above. %Therefore, we run our algorithm using this set for the remaining datasets.

\begin{comment}

\begin{figure*}[!htbp]    
\begin{adjustwidth}{-1cm}{-1cm}
\centering
\begin{subfigure}{0.35\textwidth}
\includegraphics[width=\columnwidth]{Figure/6-obj-old/R0/fig/scatter_mattrix}
\caption{R0}
%\label{fig:con_pr09}
\end{subfigure}    
\begin{subfigure}{0.35\textwidth}
\includegraphics[width=\columnwidth]{Figure/6-obj-old/R4/fig/scatter_mattrix}
\caption{R4}
%\label{fig:con_pr09}
\end{subfigure}
\begin{subfigure}{0.35\textwidth}
\includegraphics[width=\columnwidth]{Figure/6-obj-old/R9/fig/scatter_mattrix}
\caption{R9}
%\label{fig:con_pr09}
\end{subfigure}
\begin{subfigure}{0.35\textwidth}
\includegraphics[width=\columnwidth]{Figure/6-obj-old/R14/fig/scatter_mattrix}
\caption{R14}
%\label{fig:con_pr09}
\end{subfigure}
\begin{subfigure}{0.35\textwidth}
\includegraphics[width=\columnwidth]{Figure/6-obj-old/R19/fig/scatter_mattrix}
\caption{R19}
%\label{fig:con_pr09}
\end{subfigure}
\caption{\underline{100-taxon simulated dataset:} Scatter-plot matrices depicting the pairwise relationship of all objective functions on five randomly selected replicates. We turn each objective function into minimization form and then normalize using min-max technique. In each matrix, the diagonal cells show the distribution of objective values (estimated using KDE) while the non-diagonal cells show the correlation between pairs of objective functions. Each upper-diagonal cell contains the value of correlation coefficient $r$ of the corresponding pair of objective functions.}
\label{fig:new_nature_obj}
\end{adjustwidth}
\end{figure*}
\begin{figure*}[!htbp]
\centering
\small
\begin{adjustwidth}{-1cm}{-1cm}
\begin{tabular}{l||C{0.24\textwidth}|C{0.24\textwidth}|C{0.24\textwidth}|C{0.24\textwidth} }
& Entropy & GapCon & SimG & SimNG\\\hline\hline
\rotatebox[origin=c]{-90}{R0} & 
\raisebox{-.5\height}{\includegraphics[width=0.25\textwidth]{Figure/6-obj-old/precomputedInit/R0/fig/Entropy_partial_regression}} &
\raisebox{-.5\height}{\includegraphics[width=0.25\textwidth]{Figure/6-obj-old/precomputedInit/R0/fig/GapCon_partial_regression}} & 
\raisebox{-.5\height}{\includegraphics[width=0.25\textwidth]{Figure/6-obj-old/precomputedInit/R0/fig/SimG_partial_regression}} & 
\raisebox{-.5\height}{\includegraphics[width=0.25\textwidth]{Figure/6-obj-old/precomputedInit/R0/fig/SimNG_partial_regression}}     
\\\hline
\rotatebox[origin=c]{-90}{R4} &
\raisebox{-.5\height}{\includegraphics[width=0.25\textwidth]{Figure/6-obj-old/precomputedInit/R4/fig/Entropy_partial_regression}} &
\raisebox{-.5\height}{\includegraphics[width=0.25\textwidth]{Figure/6-obj-old/precomputedInit/R4/fig/GapCon_partial_regression}} & 
\raisebox{-.5\height}{\includegraphics[width=0.25\textwidth]{Figure/6-obj-old/precomputedInit/R4/fig/SimG_partial_regression}} & 
\raisebox{-.5\height}{\includegraphics[width=0.25\textwidth]{Figure/6-obj-old/precomputedInit/R4/fig/SimNG_partial_regression}}
\\\hline
\rotatebox[origin=c]{-90}{R9} &
\raisebox{-.5\height}{\includegraphics[width=0.25\textwidth]{Figure/6-obj-old/precomputedInit/R9/fig/Entropy_partial_regression}} &
\raisebox{-.5\height}{\includegraphics[width=0.25\textwidth]{Figure/6-obj-old/precomputedInit/R9/fig/GapCon_partial_regression}} & 
\raisebox{-.5\height}{\includegraphics[width=0.25\textwidth]{Figure/6-obj-old/precomputedInit/R9/fig/SimG_partial_regression}} & 
\raisebox{-.5\height}{\includegraphics[width=0.25\textwidth]{Figure/6-obj-old/precomputedInit/R9/fig/SimNG_partial_regression}}
\\\hline
\rotatebox[origin=c]{-90}{R14} &
\raisebox{-.5\height}{\includegraphics[width=0.25\textwidth]{Figure/6-obj-old/precomputedInit/R14/fig/Entropy_partial_regression}} &
\raisebox{-.5\height}{\includegraphics[width=0.25\textwidth]{Figure/6-obj-old/precomputedInit/R14/fig/GapCon_partial_regression}} & 
\raisebox{-.5\height}{\includegraphics[width=0.25\textwidth]{Figure/6-obj-old/precomputedInit/R14/fig/SimG_partial_regression}} & 
\raisebox{-.5\height}{\includegraphics[width=0.25\textwidth]{Figure/6-obj-old/precomputedInit/R14/fig/SimNG_partial_regression}}
\\\hline
\rotatebox[origin=c]{-90}{R19} &
\raisebox{-.5\height}{\includegraphics[width=0.25\textwidth]{Figure/6-obj-old/precomputedInit/R19/fig/Entropy_partial_regression}} &
\raisebox{-.5\height}{\includegraphics[width=0.25\textwidth]{Figure/6-obj-old/precomputedInit/R19/fig/GapCon_partial_regression}} & 
\raisebox{-.5\height}{\includegraphics[width=0.25\textwidth]{Figure/6-obj-old/precomputedInit/R19/fig/SimG_partial_regression}} & 
\raisebox{-.5\height}{\includegraphics[width=0.25\textwidth]{Figure/6-obj-old/precomputedInit/R19/fig/SimNG_partial_regression}}
\\\hline
\end{tabular}    
\caption{\underline{100-taxon simulated dataset:} Multiple linear regression model for identifying the association among FN rate and three objective functions (SimNG, GapCon and SimG/Entropy) fitted to five randomly selected replicates. There is one figure for each possible combination (replicate, objective function). Each partial regression plot shows the association between an objective function and FN rate while holding the remaining two objectives constant.  In a plot for an objective function $ OF $, the horizontal axis, $e(OF|X)$, denotes the residuals from regressing $OF$ against the remaining objective functions and the vertical axis, $e(FNRate|X)$, denotes the residuals from regressing FN rate against all the objective functions except $ OF $.}
\label{fig:new_mul_lin_reg}
\end{adjustwidth}
\end{figure*}
\end{comment}

\subsubsection{Selection of a new formulation}
\label{sec:new_msa_formulation}
The results reported in the last subsection suggest that the objective functions that exhibit a good association with FN rate should be more effective than the other objective functions for estimating phylogenetic trees. Based on this we make an attempt to form a new objective set as follows. We first propose four new objective functions that quantify different aspects of MSA: Entropy, SimG, SimNG and GapCon (the details are presented in Section~\ref{sec:formulation}). We combine these with TC and Gap and run NSGA-III to optimize the objective set \{Entropy, TC, Gap, SimG, SimNG, GapCon\} for 40 times to generate numerous diverse alignments. We used those to examine the association of our proposed objective functions with FN rate using multiple linear regression analysis (please refer to Figure~\ref{fig:new_nature_obj} of the supplementary file for a visualization of the relationship between the relevant pairs of objective functions within the set). The key observations of this analysis are summarized as follows: 

\begin{itemize}
	\item Entropy has a strong correlation with SimG which is problematic for multiple regression analysis. So, we should not keep these two objectives at the same time in our regression model as well as in the multi-objective formulation.
	
	\item SimG and SimNG are in conflict with each other. So by optimizing them simultaneously, a multi-objective metaheuristic can generate a large number of diverse alignments.
\end{itemize}

Now we express the relationship between FN rate and the proposed objective functions using the following model:
\begin{equation}
\small
\begin{split}
\text{FN rate} = \beta_0 + \beta_1 \times \text{SimNG}+ \beta_2 \times \text{GapCon} + \\
\beta_3 \times \text{SimG (or Entropy)} + \epsilon \label{eq:new_multi_lin_reg}
\end{split}
\end{equation}

We estimate the regression coefficients by fitting the above model to the solutions generated by optimizing the objective set \{Entropy, TC, Gap, SimG, SimNG, GapCon\}. Here we find that, in each case, both SimG and SimNG exhibit a positive correlation with FN rate. So we choose \{SimG, SimNG\} as our new objective set. For a visualization of the results using partial regression plots please refer to Figure~\ref{fig:new_mul_lin_reg} of the supplementary file.


%\clearpage
\subsection{Validation of the selected multi-objective formulations}%{Results on biological rRNA datasets}
Now we examine the effectiveness of our chosen formulations (i.e. \{Gap, SOP\} and \{SimG, SimNG\}) based on two biological datasets: two biological rRNA datasets and 27 instances of the BAliBASE 3.0 benchmark. To accomplish this we conduct several independent runs of NSGA-II on each dataset considering its stochastic nature according to the standard practice of OR literature. Then we analyze the generated solutions based on the quality of the generated alignments as well as the resultant trees.

\subsubsection{Results on biological rRNA datasets}
\begin{figure*}[!htbp]
	\centering
	\begin{adjustwidth}{-0.5cm}{-0.5cm}
		\begin{subfigure}[b]{0.25\textwidth}
			\includegraphics[width=\columnwidth]{Figure/summary/precomputedInit/23S.E/fnrate_density_single_run}
			\caption{23S.E}
			%\label{fig:con_pr09}
		\end{subfigure}    
		\begin{subfigure}[b]{0.25\textwidth}
			\includegraphics[width=\columnwidth]{Figure/summary/precomputedInit/23S.E.aa_ag/fnrate_density_single_run}
			\caption{23S.E.aa\_ag}
			%\label{fig:con_pr09}
		\end{subfigure}
		\begin{subfigure}[b]{0.25\textwidth}
			\includegraphics[width=\columnwidth]{Figure/summary/precomputedInit/23S.E/objset_fnrate_rank}
			\caption{23S.E}
			%\label{fig:con_pr09}
		\end{subfigure}    
		\begin{subfigure}[b]{0.25\textwidth}
			\includegraphics[width=\columnwidth]{Figure/summary/precomputedInit/23S.E.aa_ag/objset_fnrate_rank}
			\caption{23S.E.aa\_ag}
			%\label{fig:con_pr09}
		\end{subfigure}
		%Figure 9 starts############################
		\begin{subfigure}{0.25\textwidth}
			\includegraphics[width=\columnwidth]{Figure/summary/precomputedInit/23S.E/tc_density_single_run}
			\caption{23S.E}
			%\label{fig:con_pr09}
		\end{subfigure}    
		\begin{subfigure}{0.25\textwidth}
			\includegraphics[width=\columnwidth]{Figure/summary/precomputedInit/23S.E.aa_ag/tc_density_single_run}
			\caption{23S.E.aa\_ag}
			%\label{fig:con_pr09}
		\end{subfigure}
		\begin{subfigure}{0.25\textwidth}
			\includegraphics[width=\columnwidth]{Figure/summary/precomputedInit/23S.E/objset_tc_rank}
			\caption{23S.E}
			%\label{fig:con_pr09}
		\end{subfigure}    
		\begin{subfigure}{0.25\textwidth}
			\includegraphics[width=\columnwidth]{Figure/summary/precomputedInit/23S.E.aa_ag/objset_tc_rank}
			\caption{23S.E.aa\_ag}
			%\label{fig:con_pr09}
		\end{subfigure}
		%Figur1 10 ##################
		\begin{subfigure}{0.25\textwidth}
			\includegraphics[width=\columnwidth]{Figure/summary/precomputedInit/23S.E/pairs_density_single_run}
			\caption{23S.E}
			%\label{fig:con_pr09}
		\end{subfigure}    
		\begin{subfigure}{0.25\textwidth}
			\includegraphics[width=\columnwidth]{Figure/summary/precomputedInit/23S.E.aa_ag/pairs_density_single_run}
			\caption{23S.E.aa\_ag}
			%\label{fig:con_pr09}
		\end{subfigure}
		\begin{subfigure}{0.25\textwidth}
			\includegraphics[width=\columnwidth]{Figure/summary/precomputedInit/23S.E/objset_pairs_rank}
			\caption{23S.E}
			%\label{fig:con_pr09}
		\end{subfigure}    
		\begin{subfigure}{0.25\textwidth}
			\includegraphics[width=\columnwidth]{Figure/summary/precomputedInit/23S.E.aa_ag/objset_pairs_rank}
			\caption{23S.E.aa\_ag}
			%\label{fig:con_pr09}
		\end{subfigure}
		%Figure 11###############
		\begin{subfigure}{0.26\textwidth}
			\includegraphics[width=\columnwidth]{Figure/summary/precomputedInit/23S.E/fnrate_vs_tc}
			\caption{23S.E}
			%\label{fig:con_pr09}
		\end{subfigure}    
		\begin{subfigure}{0.26\textwidth}
			\includegraphics[width=\columnwidth]{Figure/summary/precomputedInit/23S.E.aa_ag/fnrate_vs_tc}
			\caption{23S.E.aa\_ag}
			%\label{fig:con_pr09}
		\end{subfigure}
		\begin{subfigure}{0.26\textwidth}
			\includegraphics[width=\columnwidth]{Figure/summary/precomputedInit/23S.E/fnrate_vs_sp}
			\caption{23S.E}
			%\label{fig:con_pr09}
		\end{subfigure}    
		\begin{subfigure}{0.26\textwidth}
			\includegraphics[width=\columnwidth]{Figure/summary/precomputedInit/23S.E.aa_ag/fnrate_vs_sp}
			\caption{23S.E.aa\_ag}
			%\label{fig:con_pr09}
		\end{subfigure}
	
	\caption{\underline{Biological rRNA datasets:} \textbf{Panel 1 (Top Panel):} part (a) and part (b) show the averaged FN rate of 100 solutions over 10 independent runs. 
		%Since each run generates 100 solutions, we make the average meaningful by sorting the 100 FN rates per run. Then we average the best FN rates obtained across all the runs. The same applies to the second best ones and so on. 
		part (c) and part (d) show the variation of the best FN rates obtained across 10 runs using boxplots.  %In all figures, we show the performance of nine state-of-the-art tools using dashed horizontal lines.
		\textbf{Panel 2 (Panel 3):} part (e) and part (f) (part (i) and part (j)) show the TC score (SP score) of 100  solutions averaged over 10 runs. 
		%At first, we sort the TC scores (SP scores) of each solution set. Then we average the TC scores (SP score) at each sorted position of all the sets. 
		part (g) and part (h) (part (k) and part (l)) show the distribution of the best TC scores (SP scores) collected from all runs.
		%Panel 2: part (e) and part (f) show the TC score of 100  solutions averaged over 10 runs. At first, we sort the TC scores of each solution set. Then we average the TC scores at each sorted position of all the sets. part (g) and part (h) show the distribution of the best TC scores collected from all runs.\\ 
		%In each figure, The horizontal lines show the performance of the state-of-the-art tools.
		%Panel 3: part (i) and part (j) show the SP score of 100 solutions averaged over 10 runs. At first, we sort the SP scores of each solution set. Then we average the SP scores at each sorted position of all the sets. part (k) and part (l) show the distribution of the best SP scores collected from all runs.\\ 
		%In each figure, The horizontal lines show the performance of the state-of-the-art tools.
		\textbf{Panel 4 (Bottom panel):} part (m) and part (n) show the relationship between FN rate and TC score for different alignments. part (o) and part (p) show the relationship between FN rate and SP score. In all panels except for the bottom one, we show the performance of nine state-of-the-art tools using dashed horizontal lines; the horizontal lines at the bottom panel mark the FN rates achieved by those tools.
	}
	\label{fig:fn_rate_tc_sp_bio}
\end{adjustwidth}
\end{figure*}


We conducted 10 runs of NSGA-II with our two selected objective sets for two datasets, namely, 23S.E and 23S.E.aa\_ag. 
We compare the performance of the multi-objective formulations with respect to FN rate against the nine state-of-the-art tools from two perspectives in the top panel of Figure~\ref{fig:fn_rate_tc_sp_bio}. Here, part (a) and (b) show the averaged FN rate of 100 solutions over 10 runs. Since each run generates 100 solutions, we make the average meaningful by sorting the 100 FN rates per run. Then we average the best FN rates across all the runs. The same applies to the second best ones and so on. These figures (part (a) and (b)) demonstrate a promising aspect of multi-objective approach that for each data it can generate a substantial number of solutions that are better than the outputs of state-of-the-art tools. However, a practitioner would be interested in the best solutions. Therefore, we summarize the variation of the best FN rate (among 100 values) across 10 runs in part (c) and (d). We see that, for both of the datasets, FSA yielded the best performance among the nine tools, followed by PRANK for 23.S.E and PASTA for 23S.E.aa\_ag. The two multi-objective formulations helped to achieve better FN rates than FSA for 23S.E.aa\_ag (part (b) and (d)). Here, on average, \{Gap ,SOP\} generates around 10\% solutions that are better than FSA and 40\% solutions that are better than PASTA as shown in part (b). On the other hand, on average \{SimG, SimNG\} produces very few solutions that are better than FSA but around 40\% solutions that are better than PASTA. Part (d) shows that \{Gap, SOP\} consistently outperforms the best tool (FSA) whereas \{SimG, SimNG\} outperforms FSA in nearly 40\% of the total runs. Now let us see the results for 23S.E (part (a) and (c)), where both of the objective sets remain between the best (FSA) and second best (PRANK) tool. Both of them generates around 5\% solutions better than PRANK.
%and achieves around 30\% improvement compared to FSA.
%We visualize the distribution of the best FN rates collected from each run using boxplots in Figure~\ref{fig:fn_rate_bio}. In that figure, we also show the FN rates of 100 solutions. We see that for 23S.E.aa\_ag, both of the sets achieved FN rates better than the best tool FSA but \{Gap ,SOP\} is more consistent in generating good results. On the other hand, for 23S.E both the sets perform between the first and second best tool which is FSA and PRANK respectively. Here \{SimG, SimNG\} performs consistently better than the other.

We perform similar analysis based on the widely used two alignment quality measures, namely, TC score and SP score and report the results in Panel 2 and 3 of Figure~\ref{fig:fn_rate_tc_sp_bio} respectively. We notice that, according to these two popular measures, for both the datasets, the alignments generated by multi-objective formulations failed to beat the best performing tool, PASTA. The clear disagreement between FN rate and TC score (part (m) and (n)) as well as between FN rate and SP score (part (o) and (p)) has been illustrated in Panel 4 of Figure~\ref{fig:fn_rate_tc_sp_bio}. To summarize, from the analysis presented in Panel 4 (bottom panel), we realize that the tools/approaches achieving better performance than our multi-objective formulations in terms of the popular measures, namely, TC score and SP score fail to achieve better FN rates than our multi-objective formulations. To elaborate, according to TC score, PASTA is the best performer among the nine tools, and our objective sets have generated several alignments having worse (lower) TC score than PASTA (and FSA). However, those alignments can produce phylogenetic trees with better FN rates than those tools. Even from among the tools, there is disagreement between TC score and FN rate: PASTA is in fact behind FSA in terms of the latter. Similarly in part (o) and (p) of Panel 4 which is dedicated to the comparison between FN rate and SP score, we find several alignments generated by the multi-objective formulations that are worse than PASTA in terms of SP score, but, achieve better FN rates than that tool. 

%In Figure~\ref{fig:tc_bio}, we show similar results as above but based on the widely used alignment quality measure TC score. We notice that according to TC, the multi-objective formulations can hardly outperform the best tool PASTA for both of the datasets. However, interestingly our multi-objective formulations produced better phylogenetic trees. Therefore, the tool which produces better TC score does not necessarily produce better FN rate which is unexpected. We illustrate this disagreement between FN rate and TC score in the top panel of Figure~\ref{fig:fnrate_vs_tc_bio} where we plot the FN rate and TC score of the tools as well as best alignments generated by each objective set. We see that according to TC score, PASTA is the best tool while FSA performs the best based on FN rate. Moreover, our objective sets generated several alignments having worse (lower) TC score than PASTA and FSA. However, those alignments can produce phylogenetic trees with better FN rate than those tools.

%Finally, we perform similar analysis based on SP score as shown in Figure~\ref{fig:sp_bio}. Again, here we find a similar disagreement between SP score and FN rate. For both the datasets, the alignments generated by the objective sets failed to beat the best tool PASTA according to SP score. We illustrate this phenomenon in the bottom panel of Figure~\ref{fig:fnrate_vs_tc_bio}, where we find several alignments (generated by the multi-objective formulations) worse than PASTA (the best tool w.r.t. SP-score) achieve better FN rates than that tool.  


%\begin{comment}

%############################# RV11
\begin{figure*}[!htbp]
	\begin{adjustwidth}{-1cm}{-1cm}
		\centering
		\begin{subfigure}[b]{0.26\textwidth}
			\includegraphics[width=\columnwidth]{Figure/summary/precomputedInit/Balibase/BB11005_fnrate_density_single_run}
			\caption{BB11005}
			%\label{fig:con_pr09}
		\end{subfigure}    
		\begin{subfigure}[b]{0.26\textwidth}
			\includegraphics[width=\columnwidth]{Figure/summary/precomputedInit/Balibase/BB11018_fnrate_density_single_run}
			\caption{BB11018}
			%\label{fig:con_pr09}
		\end{subfigure}
		\begin{subfigure}[b]{0.26\textwidth}
			\includegraphics[width=\columnwidth]{Figure/summary/precomputedInit/Balibase/BB11020_fnrate_density_single_run}
			\caption{BB11020}
			%\label{fig:con_pr09}
		\end{subfigure}
		\begin{subfigure}[b]{0.26\textwidth}
			\includegraphics[width=\columnwidth]{Figure/summary/precomputedInit/Balibase/BB11033_fnrate_density_single_run}
			\caption{BB11033}
			%\label{fig:con_pr09}
		\end{subfigure}
		%    \begin{subfigure}{0.26\textwidth}
		%        \includegraphics[width=\columnwidth]{Figure/summary/precomputedInit/Balibase/BB11038_fnrate_density_single_run}
		%        \caption{BB11038}
		%        %\label{fig:con_pr09}
		%    \end{subfigure}
		\begin{subfigure}{0.26\textwidth}
			\includegraphics[width=\columnwidth]{Figure/summary/precomputedInit/Balibase/BB11005_objset_fnrate_rank}
			\caption{BB11005}
			%\label{fig:con_pr09}
		\end{subfigure}    
		\begin{subfigure}{0.26\textwidth}
			\includegraphics[width=\columnwidth]{Figure/summary/precomputedInit/Balibase/BB11018_objset_fnrate_rank}
			\caption{BB11018}
			%\label{fig:con_pr09}
		\end{subfigure}
		\begin{subfigure}{0.26\textwidth}
			\includegraphics[width=\columnwidth]{Figure/summary/precomputedInit/Balibase/BB11020_objset_fnrate_rank}
			\caption{BB11020}
			%\label{fig:con_pr09}
		\end{subfigure}
		%        \begin{subfigure}{0.26\textwidth}
		%            \includegraphics[width=\columnwidth]{Figure/summary/precomputedInit/Balibase/BB11029_objset_fnrate_rank}
		%            \caption{BB11029}
		%            %\label{fig:con_pr09}
		%        \end{subfigure}
		\begin{subfigure}{0.26\textwidth}
			\includegraphics[width=\columnwidth]{Figure/summary/precomputedInit/Balibase/BB11033_objset_fnrate_rank}
			\caption{BB11033}
			%\label{fig:con_pr09}
		\end{subfigure}
		
		%Fig 15
		\begin{subfigure}{0.26\textwidth}
			\includegraphics[width=\columnwidth]{Figure/summary/precomputedInit/Balibase/BB11005_fnrate_vs_tc_2}
			\caption{BB11005}
			%\label{fig:con_pr09}
		\end{subfigure}    
		\begin{subfigure}{0.26\textwidth}
			\includegraphics[width=\columnwidth]{Figure/summary/precomputedInit/Balibase/BB11018_fnrate_vs_tc_2}
			\caption{BB11018}
			%\label{fig:con_pr09}
		\end{subfigure}
		\begin{subfigure}{0.26\textwidth}
			\includegraphics[width=\columnwidth]{Figure/summary/precomputedInit/Balibase/BB11020_fnrate_vs_tc_2}
			\caption{BB11020}
			%\label{fig:con_pr09}
		\end{subfigure}
		\begin{subfigure}{0.26\textwidth}
			\includegraphics[width=\columnwidth]{Figure/summary/precomputedInit/Balibase/BB11033_fnrate_vs_tc_2}
			\caption{BB11033}
			%\label{fig:con_pr09}
		\end{subfigure}    
		\begin{subfigure}{0.26\textwidth}
			\includegraphics[width=\columnwidth]{Figure/summary/precomputedInit/Balibase/BB11005_fnrate_vs_sp_2}
			\caption{BB11005}
			%\label{fig:con_pr09}
		\end{subfigure}    
		\begin{subfigure}{0.26\textwidth}
			\includegraphics[width=\columnwidth]{Figure/summary/precomputedInit/Balibase/BB11018_fnrate_vs_sp_2}
			\caption{BB11018}
			%\label{fig:con_pr09}
		\end{subfigure}
		\begin{subfigure}{0.26\textwidth}
			\includegraphics[width=\columnwidth]{Figure/summary/precomputedInit/Balibase/BB11020_fnrate_vs_sp_2}
			\caption{BB11020}
			%\label{fig:con_pr09}
		\end{subfigure}
		\begin{subfigure}{0.26\textwidth}
			\includegraphics[width=\columnwidth]{Figure/summary/precomputedInit/Balibase/BB11033_fnrate_vs_sp_2}
			\caption{BB11033}
			%\label{fig:con_pr09}
		\end{subfigure}
	
	\caption{ \underline{RV11:} \textbf{Panel 1 (Top panel):} part (a) - (d) show the FN rate of 100 solutions averaged over 20 independent runs. 
	%At first, we sort the FN rates of each solution set. Then we average the FN rates at each sorted position of all the sets. 
	\textbf{Panel 2:} part (e) - (h) show the distribution of the best FN rates collected from all runs. 
	\textbf{Panel 3 (Panel 4):} part (i) - (l) (part (m) - (p)) show the relationship between FN rate and TC score (SP score) for different alignments. In all panels, we show the FN rates achieved by the nine state-of-the-art tools using dashed horizontal lines.}
	\label{fig:rv11_fn_rate_tc_sp}
\end{adjustwidth}
\end{figure*}


\begin{figure*}[!htbp]
	\begin{adjustwidth}{-1cm}{-1cm}
		\centering
		\begin{subfigure}[b]{0.26\textwidth}
			\includegraphics[width=\columnwidth]{Figure/summary/precomputedInit/Balibase/BB11005_tc_density_single_run_2}
			\caption{BB11005}
			%\label{fig:con_pr09}
		\end{subfigure}    
		\begin{subfigure}[b]{0.26\textwidth}
			\includegraphics[width=\columnwidth]{Figure/summary/precomputedInit/Balibase/BB11018_tc_density_single_run_2}
			\caption{BB11018}
			%\label{fig:con_pr09}
		\end{subfigure}
		\begin{subfigure}[b]{0.26\textwidth}
			\includegraphics[width=\columnwidth]{Figure/summary/precomputedInit/Balibase/BB11020_tc_density_single_run_2}
			\caption{BB11020}
			%\label{fig:con_pr09}
		\end{subfigure}
		%        \begin{subfigure}{0.22\textwidth}
		%            \includegraphics[width=\columnwidth]{Figure/summary/precomputedInit/Balibase/BB11029_tc_density_single_run_2}
		%            \caption{BB11029}
		%            %\label{fig:con_pr09}
		%        \end{subfigure}
		\begin{subfigure}[b]{0.26\textwidth}
			\includegraphics[width=\columnwidth]{Figure/summary/precomputedInit/Balibase/BB11033_tc_density_single_run_2}
			\caption{BB11033}
			%\label{fig:con_pr09}
		\end{subfigure}
		\begin{subfigure}{0.26\textwidth}
			\includegraphics[width=\columnwidth]{Figure/summary/precomputedInit/Balibase/BB11005_objset_tc_rank_2}
			\caption{BB11005}
			%\label{fig:con_pr09}
		\end{subfigure}    
		\begin{subfigure}{0.26\textwidth}
			\includegraphics[width=\columnwidth]{Figure/summary/precomputedInit/Balibase/BB11018_objset_tc_rank_2}
			\caption{BB11018}
			%\label{fig:con_pr09}
		\end{subfigure}
		\begin{subfigure}{0.26\textwidth}
			\includegraphics[width=\columnwidth]{Figure/summary/precomputedInit/Balibase/BB11020_objset_tc_rank_2}
			\caption{BB11020}
			%\label{fig:con_pr09}
		\end{subfigure}
		%        \begin{subfigure}{0.22\textwidth}
		%            \includegraphics[width=\columnwidth]{Figure/summary/precomputedInit/Balibase/BB11029_objset_tc_rank_2}
		%            \caption{BB11029}
		%            %\label{fig:con_pr09}
		%        \end{subfigure}
		\begin{subfigure}{0.26\textwidth}
			\includegraphics[width=\columnwidth]{Figure/summary/precomputedInit/Balibase/BB11033_objset_tc_rank_2}
			\caption{BB11033}
			%\label{fig:con_pr09}
		\end{subfigure}
		
		%Fig 14
		\begin{subfigure}{0.26\textwidth}
			\includegraphics[width=\columnwidth]{Figure/summary/precomputedInit/Balibase/BB11005_pairs_density_single_run_2}
			\caption{BB11005}
			%\label{fig:con_pr09}
		\end{subfigure}    
		\begin{subfigure}{0.26\textwidth}
			\includegraphics[width=\columnwidth]{Figure/summary/precomputedInit/Balibase/BB11018_pairs_density_single_run_2}
			\caption{BB11018}
			%\label{fig:con_pr09}
		\end{subfigure}
		\begin{subfigure}{0.26\textwidth}
			\includegraphics[width=\columnwidth]{Figure/summary/precomputedInit/Balibase/BB11020_pairs_density_single_run_2}
			\caption{BB11020}
			%\label{fig:con_pr09}
		\end{subfigure}
		%        \begin{subfigure}{0.22\textwidth}
		%            \includegraphics[width=\columnwidth]{Figure/summary/precomputedInit/Balibase/BB11029_pairs_density_single_run_2}
		%            \caption{BB11029}
		%            %\label{fig:con_pr09}
		%        \end{subfigure}
		\begin{subfigure}{0.26\textwidth}
			\includegraphics[width=\columnwidth]{Figure/summary/precomputedInit/Balibase/BB11033_pairs_density_single_run_2}
			\caption{BB11033}
			%\label{fig:con_pr09}
		\end{subfigure}
		\begin{subfigure}{0.26\textwidth}
			\includegraphics[width=\columnwidth]{Figure/summary/precomputedInit/Balibase/BB11005_objset_pairs_rank_2}
			\caption{BB11005}
			%\label{fig:con_pr09}
		\end{subfigure}    
		\begin{subfigure}{0.26\textwidth}
			\includegraphics[width=\columnwidth]{Figure/summary/precomputedInit/Balibase/BB11018_objset_pairs_rank_2}
			\caption{BB11018}
			%\label{fig:con_pr09}
		\end{subfigure}
		\begin{subfigure}{0.26\textwidth}
			\includegraphics[width=\columnwidth]{Figure/summary/precomputedInit/Balibase/BB11020_objset_pairs_rank_2}
			\caption{BB11020}
			%\label{fig:con_pr09}
		\end{subfigure}
		%        \begin{subfigure}{0.22\textwidth}
		%            \includegraphics[width=\columnwidth]{Figure/summary/precomputedInit/Balibase/BB11029_objset_pairs_rank_2}
		%            \caption{BB11029}
		%            %\label{fig:con_pr09}
		%        \end{subfigure}
		\begin{subfigure}{0.26\textwidth}
			\includegraphics[width=\columnwidth]{Figure/summary/precomputedInit/Balibase/BB11033_objset_pairs_rank_2}
			\caption{BB11033}
			%\label{fig:con_pr09}
		\end{subfigure}
	\end{adjustwidth}
	\caption{\underline{RV11:} \textbf{Panel 1 (Panel 3)}: part (a) - (d) (part (i) - (l)) shows the TC score (SP score) of 100 solutions averaged over 20 runs. 
	%At first, we sort the TC scores (SP scores) of each solution set. Then we average the TC scores (Sp scores) at each sorted position of all the sets. 
	\textbf{Panel 2 (Panel 4)}: part (e) - (h) (part (m) - (p)) shows the distribution of the best TC scores (SP scores) collected from all runs. In all panels, we show the performance of nine state-of-the-art tools using dashed horizontal lines.}
	\label{fig:rv11_tc_sp}
	
\end{figure*}


\subsubsection{Results on BAliBASE datasets}
For each of the selected BAliBASE datasets under six groups (RV11, RV12, RV20, RV30, RV40 and RV50), we conducted 20 independent runs of NSGA-II. Once again we analyze the generated solutions based on the quality of alignments and resultant trees. We witnessed that the alignments that are better according to widely accepted alignment scores, not necessarily generate better phylogenetic trees.
%As the results for all the groups are very similar and consistent with our previous findings, in this section we present the results for RV11 and RV12. The results for the other groups can be found in the supplementary file.
Here we discuss our key observations on the selected four datasets (BB11005, BB11018, BB11020 and BB11033) under the group RV11. For the remaining groups (RV12, RV20, RV30, RV40 and RV50), our obtained results are similar. For the sake of brevity, we present those results in Section~\ref{sec:result_balibase} of the supplementary file. Figures \ref{fig:rv11_fn_rate_tc_sp} and \ref{fig:rv11_tc_sp} present the results of our experiments on the datasets of group RV11. According to FN rate (part (a) - (h) of Figure \ref{fig:rv11_fn_rate_tc_sp}), at least one of the two objective sets generates better or equivalent solutions than the best tool throughout all the instances. For BB11020, \{SimG, SimNG\} can achieve 12\% FN rate as opposed to 50\% FN rate attained by the best tool which is a huge improvement. Considering TC score (part (a) - (h) of Figure \ref{fig:rv11_tc_sp}), the two objective sets can outperform all the tools only for BB11020 which is contrary to the findings based on FN rate. So again we see the disagreement between FN rate and TC score which we examine graphically in part (i) - (l) of Figure~\ref{fig:rv11_fn_rate_tc_sp}. If we observe the results based on SP score (part (i) - (p) of Figure \ref{fig:rv11_tc_sp}), we get similar disagreement between FN rate and SP score which is illustrated in part (m) - (p) of Figure~\ref{fig:rv11_fn_rate_tc_sp}. These figures provide evidence that a solution with the best TC and/or SP score may not give the best FN rate. We consistently observe this phenomenon across the remaining datasets as well which we present in Section~\ref{sec:result_balibase} of the supplementary file.

Table~\ref{tab:balibase_good_solutions} shows a comparative summary of the 100 solutions generated by a single run of NSGA-II while optimizing \{Gap, SOP\} with respect to the nine state-of-the-art MSA tools based on FN rate for the 27 randomly selected BAliBASE datasets. Here we see that the multi-objective formulation has been able to generate better phylogenetic trees than all the state-of-the-art MSA tools except on a few cases (marked by cells with 0 value).

%. These are relatively small datasets with 8-14 taxa. And each of the two objective set performs better than the other for two cases. 

\begin{table}[htbp]
	%\small
	%\begin{adjustwidth}{-0.7cm}{0cm}
	\centering
	\caption{Comparative summary of the 100 solutions generated by a single run of NSGA-II while optimizing \{Gap, SOP\} with respect to the nine state-of-the-art MSA tools based on FN rate.}
	\tabcolsep=0.10cm
	\begin{tabular}{|c|l|r|r|r|r|r|r|r|r|r|}
		\hline
		\multirow{2}{*}{Group} & \multicolumn{1}{c|}{\multirow{2}{*}{Dataset}} & \multicolumn{9}{c|}{\makecell{Avg. no. of solutions (out of 100) generated by \\a single run of NSGA-II which are\\ \textbf{better or equivalent} to an MSA tool}} \\
		\cline{3-11}          &       & \rotatebox[origin=c]{90}{T-Coffee} & \rotatebox[origin=c]{90}{Clustal W} & \rotatebox[origin=c]{90}{FSA} & \rotatebox[origin=c]{90}{Kalign} & \rotatebox[origin=c]{90}{MAFFT} & \rotatebox[origin=c]{90}{MUSCLE} & \rotatebox[origin=c]{90}{PASTA} & \rotatebox[origin=c]{90}{ProbCons} & \rotatebox[origin=c]{90}{RetAlign} \\
		\hline
		\multirow{4}{*}{RV11} & BB11005 & 100   & 2     & 100   & 100   & 54    & 54    & 86    & 86    & 86 \\
		\cline{2-11}          & BB11018 & 36    & 64    & 100   & 86    & 36    & 86    & 97    & 17    & 17 \\
		\cline{2-11}          & BB11033 & 71    & 71    & 97    & 71    & 71    & 0     & 9     & 71    & 9 \\
		\cline{2-11}          & BB11020 & 34    & 34    & 100   & 100   & 34    & 0     & 0     & 0     & 34 \\
		\hline
		\hline
		\multirow{5}{*}{RV12} & BB12001 & 55    & 92    & 92    & 55    & 99    & 92    & 92    & 55    & 25 \\
		\cline{2-11}          & BB12013 & 9     & 100   & 9     & 9     & 9     & 9     & 9     & 9     & 9 \\
		\cline{2-11}          & BB12022 & 12    & 12    & 12    & 12    & 97    & 12    & 12    & 12    & 97 \\
		\cline{2-11}          & BB12035 & 80    & 100   & 11    & 100   & 29    & 90    & 80    & 20    & 73 \\
		\cline{2-11}          & BB12044 & 23    & 23    & 23    & 100   & 23    & 23    & 88    & 23    & 88 \\
		\hline
		\hline
		\multirow{5}{*}{RV20} & BB20001 & 92    & 0     & 1     & 0     & 0     & 92    & 0     & 0     & 15 \\
		\cline{2-11}          & BB20010 & 23    & 99    & 1     & 7     & 77    & 23    & 77    & 23    & 7 \\
		\cline{2-11}          & BB20022 & 82    & 59    & 82    & 100   & 82    & 82    & 59    & 59    & 0 \\
		\cline{2-11}          & BB20033 & 96    & 5     & 96    & 19    & 82    & 29    & 68    & 58    & 88 \\
		\cline{2-11}          & BB20041 & 57    & 71    & 38    & 22    & 30    & 83    & 65    & 51    & 71 \\
		\hline
		\hline
		\multirow{4}{*}{RV30} & BB30002 & 0     & 85    & 45    & 7     & 45    & 0     & 7     & 7     & 7 \\
		\cline{2-11}          & BB30008 & 53    & 53    & 98    & 25    & 98    & 98    & 100   & 38    & 90 \\
		\cline{2-11}          & BB30015 & 61    & 93    & 93    & 88    & 88    & 61    & 100   & 88    & 88 \\
		\cline{2-11}          & BB30022 & 64    & 19    & 47    & 0     & 2     & 47    & 19    & 5     & 84 \\
		\hline
		\hline
		\multirow{5}{*}{RV40} & BB40001 & 60    & 86    & 60    & 41    & 75    & 41    & 97    & 60    & 97 \\
		\cline{2-11}          & BB40013 & 51    & 39    & 39    & 45    & 26    & 45    & 45    & 62    & 62 \\
		\cline{2-11}          & BB40025 & 0     & 0     & 0     & 0     & 0     & 49    & 49    & 49    & 49 \\
		\cline{2-11}          & BB40038 & 26    & 90    & 66    & 15    & 15    & 66    & 0     & 2     & 66 \\
		\cline{2-11}          & BB40048 & 69    & 89    & 69    & 89    & 69    & 89    & 69    & 69    & 69 \\
		\hline
		\hline
		\multirow{4}{*}{RV50} & BB50001 & 10    & 85    & 62    & 94    & 10    & 100   & 62    & 85    & 85 \\
		%\cline{2-11}          & BB50001 & 10    & 85    & 62    & 94    & 10    & 100   & 62    & 85    & 85 \\
		\cline{2-11}          & BB50005 & 0     & 93    & 0     & 93    & 93    & 93    & 93    & 0     & 93 \\
		\cline{2-11}          & BB50010 & 0     & 0     & 0     & 0     & 0     & 28    & 0     & 0     & 96 \\
		\cline{2-11}          & BB50016 & 66    & 96    & 0     & 78    & 66    & 78    & 54    & 96    & 78 \\
		\hline
	\end{tabular}%
	\label{tab:balibase_good_solutions}%
	%\end{adjustwidth}
\end{table}%

\subsubsection{Statistical significance}
%Also, SP score (Figure \ref{fig:rv12_sp}) gives similar message more or less.
Now we confirm the significance of the improvement achieved by the multi-objective formulations in terms of FN rate over nine MSA tools on 27 BAliBASE datasets by applying an appropriate statistical test. We form paired data by picking the FN rate achieved by each (MSA method, dataset) pair. For the metaheuristics, we take the average of the 20 best FN rates from 20 independent runs considering its stochastic nature. As our data do not satisfy the condition of normality and homoscedasticity~\citep{sheskin2003handbook}, we choose a series of nonparametric tests following the recommendation of~\cite{derrac2011practical}. When applying these tests, we used the $p$-value threshold as 0.05 which is equivalent to 95\% confidence level.

At first, we simultaneously compare all the methods using the Friedman test~\citep{friedman1937use} which gives the relative ranking (lower is better) of all the methods and strongly suggests the existence of significant differences among the methods considered (as $p$-value is 0). The results have been presented in Column 2 of Table~\ref{tab:friedman_holm}. Here we see that the multi-objective formulations achieve the top two positions. 
Next, we complement the Friedman test by following Holm's post-hoc procedure~\citep{holm1979simple} to contrast the difference between the multi-objective formulations and each of the nine tools. The results have been summarized in Columns 3 and 4 of Table~\ref{tab:friedman_holm}. Here, each cell shows the adjusted $p$-value which indicates the significance of the difference in performance (based on FN rate) between two methods. We notice that all the $p$-values are very close to 0 and the values for \{SimG, SimNG\} are lower than \{Gap, SOP\}. So we can state with high confidence that, the multi-objective formulations achieve statistically significant improvement over the nine MSA tools.

% Table generated by Excel2LaTeX from sheet 'Sheet5'
\begin{table*}[htbp]
	%\small
	\centering
	\caption{\underline{Friedman test (Column 2):} The Average Friedman's ranking (lower is better) achieved by the MSA methods over 27 BAliBASE datasets. We performed the Friedman test based on FN rate achieved by the tools. 
		%For our metaheuristics based multi-objective formulations, we consider the average of the 20 best FN rates obtained from 20 runs. We also show the computed statistics and corresponding $ p $-value. 
		\underline{Holm's post-hoc procedure (Columns 3 and 4):} Comparison between the metaheuristics and the MSA tool using the Holm's post-hoc procedures (as a complement of the Friedman test) over 27 BAliBASE datasets. 
		%Each entry shows the adjusted $p$-value which indicates the significance of the difference in performance (based on FN rate) between two methods.
	}
	\begin{tabular}{|l|r||r|r|}
		\hline
		\multicolumn{1}{|c|}{1} & \multicolumn{1}{c||}{2} & \multicolumn{1}{c|}{3} & \multicolumn{1}{c|}{4} \\
		\hline
		\multicolumn{1}{|c|}{\multirow{2}{*}{Method}} &  \multirow{2}{*}{Friedman's Rank*}  & \multicolumn{2}{c|}{Holm's adjusted $p$-value} \\
		\cline{3-4}     &  & NSGA-II$_{\text{\{SimG, SimNG\}}}$ & NSGA-II$_{\text{\{Gap, SOP\}}}$ \\
		\hline
		NSGA-II$_{\text{\{SimG, SimNG\}}}$ & 2.2037 & \multicolumn{1}{c|}{~~~~~~~~~~~~~~~~~~~~-} & 0.46018 \\
		\hline
		NSGA-II$_{\text{\{Gap, SOP\}}}$ & 2.8704 & 0.46018 & \multicolumn{1}{c|}{~~~~~~~~~~~~~~~~-} \\
		\hline
		ProbCons & 5.7963 & 0.00014 & 0.00238 \\
		\hline
		Clustal $\Omega$ & 6.2963 & 0.00002 & 0.00044 \\
		\hline
		MAFFT & 6.4074 & 0.00001 & 0.00036 \\
		\hline
		Kalign & 6.7037 & 0.00000 & 0.00011 \\
		\hline
		PASTA & 6.8148 & 0.00000 & 0.00007 \\
		\hline
		FSA   & 6.9444 & 0.00000 & 0.00004 \\
		\hline
		MUSCLE & 7.1482 & 0.00000 & 0.00002 \\
		\hline
		Clustal W & 7.3519 & 0.00000 & 0.00001 \\
		\hline
		RetAlign & 7.4630 & 0.00000 & 0.00000 \\
		\hline
		\hline
		\multicolumn{1}{|c|}{*Statistic} & 10.5911 & \multicolumn{2}{c|}{\multirow{2}{*}{N/A}} \\
		\cline{1-2}    \multicolumn{1}{|c|}{*$p$-value} & 0.00000 & \multicolumn{2}{c|}{} \\
		\hline
	\end{tabular}%
	\label{tab:friedman_holm}%
\end{table*}%

\subsubsection{Running time}
As we are dealing with an offline optimization problem, the runtime is not a major concern in this study. Our multi-objective metaheuristics make an effort to generate improved MSAs for phylogeny estimation by evolving a set of candidate solutions. So depending on the size of the set of candidate solutions, our approach may exhibit higher running time than the state-of-the-art MSA tools; in fact, in our experiments, our approach does require a higher running time. Nonetheless, to put everything into context, here we report runtimes of our multi-objective approaches as well as MAFFT~\citep{katoh2002mafft} that can generate a competitive alignment within a very reasonable time~\citep{ashkenazy2018multiple}, keeping in mind that the former approach leverages some altered versions of the alignments output by the latter tools. Figure~\ref{fig:runtime_comp} summarizes the average runtimes for each group of BAliBASE datasets. It helps us to identify the differences in runtimes between a two objectives approach and a four objectives one, which would be informative to practitioners and method developers. From this figure, we see that the runtimes of the multi-objective approaches are at least 10 times higher than that of MAFFT. Overall, the set of nonparametric objectives \{SimG, SimNG\} exhibits the lowest runtime among the multi-objective approaches. In several cases (such as, RV12, RV20, RV30, RV40), \{SimG, SimNG\} runs more than 1.5 times faster than \{Gap, SOP\}. The calculation of Gap takes a longer period compared to other objectives due to the additional effort of reading the substitution table values continuously. The evaluation of objective functions have been shown in the literature~\citep{zambrano2017m2align} to be the main computational bottleneck for computing MSAs by multi-objective metaheuristics. Therefore, the inclusion of Gap as an objective can heavily affect the overall running time of any algorithm. Moreover, by comparing the runtimes of \{SimG, SimNG\} and \{Gap, SOP, SimG, SimNG\}, we find that the runtimes of multi-objective metaheuristics increase linearly in the number of objectives. And the increase in runtime of the four objectives approach is mostly due to Gap. This can encourage more research effort in this direction as adding appropriate objective would definitely increase the accuracy of a multi-objective approach.
\begin{figure}[!htbp] 
	\centering
	%\begin{adjustwidth}{-0.2cm}{-0.2cm}
	\includegraphics[width=0.4\textwidth]{Figure/balibase_runtime_comparison}
	%\vspace{-0.6cm}
	\caption{Average runtimes of multi-objective approaches and MAFFT for each group of BAliBASE datasets.} 
	\label{fig:runtime_comp}
	%\end{adjustwidth}
\end{figure}

\section{Discussion \& Conclusion}
\label{sec:discussion}
In this study, we have introduced an application-aware multi-objective formulation to compute MSAs with an ultimate goal to infer the phylogenetic tree from the resultant alignments. To optimize MSA, we proposed two simple objective functions in addition to the existing ones. We judged the potential capability of each objective function to yield better trees by employing domain knowledge as well as by applying statistical approaches. We employed multiple linear regression to measure the degree of association between the individual objective functions and the quality of the inferred phylogenetic tree (i.e., FN rate). Thus, we provide empirical justification to choose two multi-objective formulations to move forward. Afterwards, we performed extensive experimentation with both simulated and biological datasets to demonstrate the benefit of our approach. We showed that the simultaneous optimization of a set of application-aware objective functions can lead to phylogenetic trees with improved accuracy than that of the state-of-the-art MSA tools. From this finding, we would like to hypothesize that, the use of domain specific measures can aid MSA methods in other application domain as well. In the sequel we identified \{SimG, SimNG\} to be the best set of objective functions for computing MSAs with an aim to infer phylogeny, considering its overall accuracy, runtime and nonparametric nature.

MSAs are computed to serve various biological purposes including phylogeny estimation and protein structure prediction. The definition of what constitutes a true alignment can depend partly on the purpose of MSAs~\citep{warnow2017computational}. Nevertheless, regardless of the purpose, the sites within the true alignment define the ``homologies''. Therefore, homology can be based on structural features or evolutionary histories, leading to the opposing concepts of ``structural homology'' and ``evolutionary homology''~\citep{warnow2017computational}. While structural alignments are expected to be close to the true (evolutionary) alignment, convergent evolution may create conditions where the best structural alignment puts nucleotides or amino acids in the same site (thus implying homologies), even though these specific homologies are not present in the true evolutionary alignment~\citep{iantorno2014watches}. In other words, structural homology may not be identical to evolutionary homology~\citep{reeck1987homology}. In such a situation, generic metrics such as TC/SP score might not be adequate to assess the correctness of the estimated MSAs. Therefore, using more informative metrics (e.g. phylogeny as done in the study) to tailor adequate multi-objective formulation of this problem seems a promising endeavor.

Standard criteria (SP score, TC score, etc.) for assessing alignment quality are usually based on shared homology pairs (SP score) or identical columns (TC score), and do not explicitly consider a particular application domain. Mistakes in alignments that are not important with respect to an application domain may not impact the ultimate accuracy of that particular inference. For example, not all sites are significant with respect to protein structure and function prediction, and hence multiple alignments with different accuracy may lead to the same predictions~\citep{warnow2013large}. Similarly, in the context of phylogeny estimation, alignments with substantially different SP scores may lead to trees with the same accuracy~\citep{liu2009rapid}. In this study, we systematically investigate the impact of evaluation criteria of an alignment on phylogenetic tree inference problem. Our results suggest that it could be possible to develop improved MSA methods for phylogenetic analysis by carefully choosing appropriate objective functions. Moreover, in almost all existing studies on MSA, we find the researchers evaluating the effectiveness of MSA methods using some generic alignment quality measures (i.e., TC score, SP score). Contrastingly, our results revealed that optimizing those widely used measures do not necessarily lead us to the best phylogenetic tree. This finding could be an eye opener for the researchers who need to use MSA methods to address a particular application. 

Our findings and proposed multi-objective formulation can be particularly beneficial for iterative methods like SAT\'e and PASTA that iteratively co-estimate both alignment and tree. These methods obtain an initial alignment and a tree that guide each other to improved estimates in an iterative fashion. They make an effort to exploit the close association between the accuracy of an MSA and the corresponding tree in finding the output through multiple iterations from both directions. Therefore, carefully choosing an evaluation metric for an MSA with a better correlation to the tree accuracy seems likely to improve the results of these co-estimation techniques. Thus, our methodology, if adopted, may potentially have a profound positive impact on the accuracy of these iterative co-estimation techniques. Moreover, multiple ``good'' alignments from the output of the multi-objective approach can be served as alternative MSAs for several methods (such as~\citep{ashkenazy2018multiple}) which would then utilize all of them to infer phylogeny with better accuracy.

This study will encourage the scientific community to investigate various application-aware measures for computing and evaluating MSAs. This will potentially prompt more experimental studies addressing specific application domains; and ultimately will propel our understanding of MSAs and their impact in various domains in computational biology, i.e, phylogeny estimation, protein structure and function prediction, orthology prediction etc. This study will also encourage the researchers to develop new scalable MSA tools by simultaneously optimizing multiple appropriate optimization criteria. Thus, we believe that this study will pioneer new models and optimization criteria for computing MSAs -- laying a firm, broad foundation for application specific multi-objective formulation for estimating multiple sequence alignment.

We performed an extensive experimental study comprising 29 datasets of varying sizes and complexities, and our findings are consistent throughout all the datasets. Still, we acknowledge the possibility of facing a few unforeseen circumstances as follows. There might be some datasets on which our approach might not exhibit satisfactory performance. Besides, currently we did not pay any effort to improve the running time of our approach which is higher as compared to top MSA tools. However, sufficient speedup could be achieved by leveraging modern computing architectures (computer cluster, GPU, etc.). 

Formulating application-aware multi-objective formulation (application specific evaluation criteria in general) cannot be developed entirely in one study; it should evolve in response to scientific findings and systematists' feedback. This requires the active involvement of evolutionary biologists, computer scientists, systematists, and others -- leading to improved understandings of alignments and how they are related to various fields in comparative genomics.



\bibliographystyle{IEEEtran} 
\bibliography{document}

\begin{IEEEbiographynophoto}{Muhammad Ali Nayeem} received his B.Sc. and M.Sc. degrees in Computer Science and Engineering from the Department of Computer Science and Engineering (CSE), Bangladesh University of Engineering and Technology (BUET), Dhaka, Bangladesh in 2013 and 2016, respectively. 
	
%He is currently an Assistant Professor with the Department of CSE, BUET. His research interest includes metaheuristics and their application in real-life optimization problems, intelligent transportation systems, bioinformatics.
\end{IEEEbiographynophoto}

\begin{IEEEbiographynophoto}{Md. Shamsuzzoha Bayzid} is an Assistant Professor in the Department of Computer Science and Engineering (CSE) at Bangladesh University of Engineering and Technology (BUET).  He received his Ph.D. in Computer Science from the University of Texas at Austin (UT Austin) in 2016, and was supervised by Prof. Tandy Warnow. He completed the B. Sc. Engg. and M. Sc. Engg. degrees in Computer Science from BUET in 2008 and 2010, respectively. %Dr. Bayzid’s research interest includes, but is not limited to, computational biology, theory and algorithms, and machine learning.	
\end{IEEEbiographynophoto}

\begin{IEEEbiographynophoto}{Atif Hasan Rahman} is an Assistant Professor in the Department of Computer Science and Engineering (CSE) at the Bangladesh University of Engineering and Technology (BUET). He completed his Ph.D. from the University of California, Berkeley in 2015 under the supervision of Lior Pachter. His research area is bioinformatics and computational biology and is currently focusing on statistical methods for genome assembly and analysis.% He was a recipient of Fulbright Science and Technology Fellowship in 2009.
\end{IEEEbiographynophoto}

\begin{IEEEbiographynophoto}{Rifat Shahriyar} is working as an Assistant Professor at Department of Computer Science and Engineering (CSE) of Bangladesh University of Engineering and Technology (BUET). He has completed Ph.D. in April 2015 from Research School of Computer Science in Australian National University (ANU). His Ph.D. under the supervision of Steve Blackburn and Kathryn McKinley changed the way people think about reference counting and conservative garbage collection. %He has completed B.Sc. in May 2007 and M.Sc. in December 2009 from CSE, BUET. His research interests are memory management specially garbage collection, virtual machine, programming language implementation, and high performance computing.
\end{IEEEbiographynophoto}

\begin{IEEEbiographynophoto}{M. Sohel Rahman} is a Professor of the CSE department of BUET. He had previously worked as a Visiting Senior Research Fellow of King's College London. He is a Senior Member of both IEEE and ACM and a member of American Mathematical Society (AMS) and London Mathematical Society (LMS). He is also a Peer-review Associate College Member of EPSRC, UK. Dr. Rahman's research interest includes, but is not limited to, theory and algorithms, stringology, metaheuristics, applied informatics, computational biology and bioinformatics. 
	%Among his highly cited results are the work on high dimensional Knapsack problems, algorithms on different variants of sequence alignment problems, data structures for different variants of string and sequence matching and sufficient conditions for Hamiltoninicity and metaheuristics solutions for hard real-life problems in different branches of science and engineering. Dr. Rahman regularly writes reviews at Mathematical Review and ACM Computing Review.
\end{IEEEbiographynophoto}

%\EOD
\vfill

\end{document}
